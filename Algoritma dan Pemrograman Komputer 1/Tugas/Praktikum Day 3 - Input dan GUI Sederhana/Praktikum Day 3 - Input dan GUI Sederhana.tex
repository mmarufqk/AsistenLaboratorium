\documentclass[12pt]{article}
\usepackage[a4paper, margin=2cm]{geometry}
\usepackage[utf8]{inputenc}
\usepackage{indentfirst}
\usepackage{enumitem}
\usepackage{fancyhdr}
\usepackage{amsmath}
\usepackage{mdframed}
\usepackage{xcolor}
\usepackage{listings}

\lstset{
    language=Java,
    basicstyle=\ttfamily\small,
    keywordstyle=\color{blue},
    commentstyle=\color{gray},
    stringstyle=\color{red},
    numbers=left,
    numberstyle=\tiny,
    stepnumber=1,
    numbersep=5pt,
    backgroundcolor=\color{white},
    showspaces=false,
    showstringspaces=false,
    showtabs=false,
    frame=single,
    tabsize=2,
    captionpos=b,
    breaklines=true,
    breakatwhitespace=true,
    escapeinside={\%*}{*)}
}

\renewcommand{\thesection}{Soal \arabic{section}}

\pagestyle{fancy}
\fancyhf{}
\rhead{TUGAS PRAKTIKUM ALPRO 1}
\lhead{BAB 4}
\rfoot{Halaman \thepage}

\title{TUGAS PRAKTIKUM ALGORITMA DAN PEMROGRAMAN 1}
\author{}
\date{Pertemuan ke-3\\BAB 4\\Input dan GUI Sederhana\\Deadline: 06 Oktober 2025}

\begin{document}

\maketitle
\thispagestyle{empty}
\setcounter{page}{0}

\section*{Petunjuk}
\begin{itemize}
    \item Kerjakan semua soal di bawah ini dengan menggunakan bahasa Java.
    \item Berikan penjelasan singkat untuk setiap program dalam komentar kode.
    \item Program harus dapat di-compile dan di-run tanpa error.
    \item Nama file source code (.java) harus sesuai dengan nama class.
    \item Kumpulkan file source code (.java) untuk setiap program dan laporan praktikum (.pdf).
    \item Format laporan praktikum dapat dilihat di myITS Classroom.
    \item Penamaan file laporan praktikum adalah \texttt{LaporanPraktikum3\_Kelompok1\_Nama Lengkap}.pdf.
    \item Hasil pengerjaan dikumpulkan di myITS Classroom dalam satu file (.zip) dengan nama \texttt{LaporanPraktikum3\_Kelompok1\_Nama Lengkap}.zip yang berisi file source code (.java) dan laporan praktikum (.pdf).
    \item Deadline pengumpulan: \textbf{06 Oktober 2025}
\end{itemize}


\newpage

\section{Melengkapi Program Biodata Mahasiswa}
Lengkapi program berikut yang meminta input data mahasiswa dan menampilkannya dalam GUI.

\begin{lstlisting}[]
import java.util.Scanner;
import javax.swing.JOptionPane;

public class DataMahasiswa {
    public static void main(String[] args) {
        // [1] Deklarasi Scanner
        ____________ input = new ____________(System.in);
        
        System.out.print("Masukkan Nama: ");
        // [2] Baca input nama (String) 
        String nama = ____________;
        
        System.out.print("Masukkan NRP: ");
        // [3] Baca input NRP (String)
        String nrp = ____________;
        
        System.out.print("Masukkan IPK: ");
        // [4] Baca input IPK (double)
        double ipk = ____________;
        
        // [5] Format output
        String hasil = "Nama: " + ____________ + 
                      "\nNRP: " + ____________ + 
                      "\nIPK: " + ____________;
        
        // [6] Tampilkan di JOptionPane
        JOptionPane.____________(null, hasil);
    }
}
\end{lstlisting}

\section{Program Kalkulator GUI dengan BufferedReader dan JOptionPane}
Buat program kalkulator sederhana yang menggunakan \textbf{BufferedReader} untuk input bilangan dan \textbf{JOptionPane} untuk input operator dan output hasil.

\newpage
\noindent \textbf{Contoh Output:}
\begin{mdframed}[backgroundcolor=gray!15]
\begin{verbatim}
// Input di konsol
Masukkan bilangan pertama: 10
Masukkan bilangan kedua: 5

// Input operator di JOptionPane
[Input Dialog] Masukkan operator (+, -, *, /): *

// Output di JOptionPane
[Message Dialog] 10.0 * 5.0 = 50.0
\end{verbatim}
\end{mdframed}

\vspace{1cm}
\end{document}