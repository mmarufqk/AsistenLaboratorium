\documentclass[12pt]{article}
\usepackage[a4paper, margin=2cm]{geometry}
\usepackage[utf8]{inputenc}
\usepackage{indentfirst}
\usepackage{enumitem}
\usepackage{fancyhdr}
\usepackage{amsmath}
\usepackage{mdframed}
\usepackage{xcolor}
\usepackage{listings}

\lstset{
    language=Java,
    basicstyle=\ttfamily\small,
    keywordstyle=\color{blue},
    commentstyle=\color{gray},
    stringstyle=\color{red},
    numbers=left,
    numberstyle=\tiny,
    stepnumber=1,
    numbersep=5pt,
    backgroundcolor=\color{white},
    showspaces=false,
    showstringspaces=false,
    showtabs=false,
    frame=single,
    tabsize=2,
    captionpos=b,
    breaklines=true,
    breakatwhitespace=true,
    escapeinside={\%*}{*)}
}

\renewcommand{\thesection}{Soal \arabic{section}}

\pagestyle{fancy}
\fancyhf{}
\rhead{TUGAS PRAKTIKUM ALPRO 1}
\lhead{BAB 9}
\rfoot{Halaman \thepage}

\title{TUGAS PRAKTIKUM ALGORITMA DAN PEMROGRAMAN 1}
\author{}
\date{Pertemuan ke-7\\BAB 9: Fungsi/Method\\Deadline: 10 November 2024 Pukul 18:29}

\begin{document}

\maketitle
\thispagestyle{empty}
\setcounter{page}{0}

\section*{Petunjuk}
\begin{itemize}
    \item Kerjakan semua soal di bawah ini dengan menggunakan bahasa Java.
    \item Implementasikan konsep Fungsi/Method sesuai materi Modul 9.
    \item Program harus dapat di-compile dan di-run tanpa error.
    \item Nama file source code (.java) harus sesuai dengan nama class.
    \item Kumpulkan file source code (.java) untuk setiap program dan laporan praktikum (.pdf).
    \item Source code di dalam laporan wajib dilampirkan menggunakan syntax highlighter.
    \item Format laporan praktikum dapat dilihat di myITS Classroom.
    \item Penamaan file laporan praktikum adalah \texttt{LaporanPraktikum7\_Kelompok1\_Nama Lengkap}.pdf.
    \item Hasil pengerjaan dikumpulkan di myITS Classroom dalam satu file (.zip) dengan nama \texttt{LaporanPraktikum7\_Kelompok1\_Nama Lengkap}.zip yang berisi file source code (.java) dan laporan praktikum (.pdf).
    \item Deadline pengumpulan: \textbf{10 November 2024 Pukul 18:29}
\end{itemize}

\newpage
\section{Kalkulator Pola Bilangan Rekursif}

Buat program kalkulator yang dapat menghitung berbagai pola bilangan menggunakan method rekursif. Program harus memiliki menu untuk memilih jenis pola bilangan yang ingin dihitung.

\textbf{Spesifikasi Program:}
\begin{itemize}
    \item Tampilkan menu pilihan pola bilangan:
    \begin{enumerate}
        \item Fibonacci
        \item Faktorial
        \item Segitiga Pascal
        \item Keluar
    \end{enumerate}
    \item Gunakan method rekursif untuk setiap perhitungan
    \item Program berjalan sampai user memilih keluar
\end{itemize}

\textbf{Rumus Pola Bilangan:}
\begin{itemize}
    \item \textbf{Fibonacci:} $F(n) = F(n-1) + F(n-2)$ dengan $F(1)=1, F(2)=1$
    \item \textbf{Faktorial:} $n! = n \times (n-1)!$ dengan $0! = 1$
    \item \textbf{Segitiga Pascal:} $C(n,k) = C(n-1,k-1) + C(n-1,k)$ dengan $C(n,0)=1, C(n,n)=1$
\end{itemize}

\textbf{Contoh Output:}
\begin{mdframed}[backgroundcolor=gray!15]
\begin{verbatim}
=== KALKULATOR POLA BILANGAN ===
1. Fibonacci
2. Faktorial  
3. Segitiga Pascal
4. Keluar

Pilihan Anda: 1
Masukkan nilai n: 7
F(7) = 13

Pilihan Anda: 2
Masukkan nilai n: 5
5! = 120

Pilihan Anda: 3
Masukkan nilai n: 4
Baris ke-4: 1 3 3 1

Pilihan Anda: 4
Terima kasih!
\end{verbatim}
\end{mdframed}

\textbf{Requirements:}
\begin{itemize}
    \item Buat method rekursif untuk setiap pola bilangan dengan base case yang tepat
    \item Gunakan method overloading jika diperlukan untuk handling input yang berbeda
    \item Program harus dapat menangani input yang tidak valid (misal: bilangan negatif) dan menampilkan pesan error yang jelas

\end{itemize}

\newpage

\section{Petualangan Java Mencari Harta Karun}

\textbf{Java adalah seorang petualang pemberani} yang gemar \textbf{menjelajah dunia pemrograman}. Suatu hari, ia mendengar kisah tentang \textbf{sebuah harta karun legendaris} yang tersembunyi jauh di Pulau A-1. Tanpa ragu, Java memulai perjalanan panjang untuk menemukannya.

Namun, perjalanan Java tidak mudah. Ia harus \textbf{mengatur energi}, \textbf{melangkah sejauh 50 satuan}, dan terkadang mencari \textbf{petunjuk rahasia} agar tidak tersesat. Tugas kamu adalah \textbf{membantu Java menyelesaikan misinya} dengan melengkapi method yang belum diimplementasikan.

Lengkapilah program berikut dengan mengisi bagian yang ditandai. Ikuti setiap instruksi pada komentar agar petualangan Java berjalan dengan sukses!

\begin{lstlisting}
import java.util.Scanner;

public class PetualanganJava {
    private static int energi = 100;
    private static int jarak = 0;
    
    public static void main(String[] args) {
        Scanner input = new Scanner(System.in);

        System.out.println("=== PETUALANGAN JAVA ===");
        System.out.println("Energi awal: " + energi);

        // LENGKAPI 1: Panggil method tampilkanPeta()

        while (energi > 0 && jarak < 50) {
            System.out.println("\nEnergi: " + energi + " | Jarak: " + jarak + "/50");
            System.out.print("Aksi (1=Jalan, 2=Istirahat, 3=Petunjuk): ");
            int aksi = input.nextInt();

            // LENGKAPI 2: Gunakan switch-case untuk aksi
            // case 1 -> jalan() lalu tambahkan langkah ke jarak
            // case 2 -> istirahat()
            // case 3 -> tampilkan petunjuk dari cariPetunjuk()
            // default -> aksi tidak valid

            System.out.println("Energi sekarang: " + energi);

            if (energi <= 20 && energi > 0) {
                System.out.println("Energi menipis! Disarankan istirahat.");
            }

            if (energi <= 0) {
                System.out.println("\nGame Over! Energi habis.");
                break;
            }

            if (cekHarta()) {
                System.out.println("\nSELAMAT! Harta ditemukan!");
                break;
            }
        }

        input.close();
    }

    public static void tampilkanPeta() {
        // LENGKAPI 3: Cetak peta perjalanan Java
    }

    public static int jalan() {
        // LENGKAPI 4: Generate langkah 1-10
        // LENGKAPI 5: Energi berkurang 5-15, energi tidak boleh negatif
        // LENGKAPI 6: Tampilkan berapa langkah dan energi berkurang
        return 0; // ubah dengan langkah sebenarnya
    }

    public static void istirahat() {
        // LENGKAPI 7: Tambah energi 20-40, maksimal 100
        // LENGKAPI 8: Cetak pesan energi bertambah
    }

    public static String cariPetunjuk() {
        // LENGKAPI 9: Energi berkurang 10
        // LENGKAPI 10: Return petunjuk acak dari array string
        return ""; // ubah dengan petunjuk sebenarnya
    }
    
    public static boolean cekHarta() {
        // LENGKAPI 11: return true jika jarak >= 50
        return false;
    }
}
\end{lstlisting}

\newpage

\textbf{Contoh Output:}
\begin{mdframed}[backgroundcolor=gray!15]
\begin{verbatim}
=== PETUALANGAN JAVA ===
Energi awal: 100

PETA:
Start [ ]-[ ]-[ ]-[ ]-[ ]-[HARTA]
      0   10  20  30  40   50

Energi: 100 | Jarak: 0/50
Aksi (1=Jalan, 2=Istirahat, 3=Petunjuk): 1
Anda berjalan 7 langkah! (-8 energi)
Energi sekarang: 92

Energi: 92 | Jarak: 7/50
Aksi (1=Jalan, 2=Istirahat, 3=Petunjuk): 3
Petunjuk: Terus maju hingga jarak 50!
Energi sekarang: 82

SELAMAT! Harta ditemukan!
\end{verbatim}
\end{mdframed}

\textbf{Petunjuk:}
\begin{itemize}
\item Gunakan \texttt{Math.random()} untuk angka acak
\item Energi tidak boleh negatif
\item Jika energi habis, permainan langsung berakhir
\item Untuk method \texttt{cariPetunjuk()} gunakan array string yang berisi beberapa petunjuk. Konsep array akan kita bahas pada modul 10, tapi untuk sementara ini sebagai pengenalan:

\begin{lstlisting}
String[] petunjuk = {
    "Terus maju hingga jarak 50!",
    "Jangan lupa istirahat saat energi menipis!",
    "Harta karun berada di jarak 50 satuan!",
    "Setiap langkah membutuhkan energi!",
    "Gunakan aksi 'Istirahat' untuk mengisi energi!"
};
int index = (int)(Math.random() * petunjuk.length);
\end{lstlisting}
\end{itemize}

\end{document}