\documentclass[12pt]{article}
\usepackage[a4paper, margin=2cm]{geometry}
\usepackage[utf8]{inputenc}
\usepackage{indentfirst}
\usepackage{enumitem}
\usepackage{fancyhdr}
\usepackage{amsmath}
\usepackage{mdframed}
\usepackage{xcolor}

\renewcommand{\thesection}{Soal \arabic{section}}

\pagestyle{fancy}
\fancyhf{}
\rhead{TUGAS PRAKTIKUM ALPRO 1}
\lhead{BAB 2 \& 3}
\rfoot{Halaman \thepage}

\title{TUGAS PRAKTIKUM ALGORITMA DAN PEMROGRAMAN 1}
\author{}
\date{Pertemuan ke-2\\BAB 2 \& 3\\Dasar Pemrograman Java \& Variabel dan Operator\\Deadline: 29 September 2025}

\begin{document}

\maketitle
\thispagestyle{empty}
\setcounter{page}{0}

\section*{Petunjuk}
\begin{itemize}
    \item Kerjakan semua soal di bawah ini dengan menggunakan bahasa Java.
    \item Berikan penjelasan singkat untuk setiap program dalam komentar kode.
    \item Program harus dapat di-compile dan di-run tanpa error.
    \item Nama file source code (.java) harus sesuai dengan nama class.
    \item Kumpulkan file source code (.java) untuk setiap program dan laporan praktikum (.pdf).
    \item Format laporan praktikum dapat dilihat di myITS Classroom.
    \item Penamaan file laporan praktikum adalah \texttt{LaporanPraktikum2\_Kelompok1\_Nama Lengkap}.pdf.
    \item Hasil pengerjaan dikumpulkan di myITS Classroom dalam satu file (.zip) dengan nama \texttt{LaporanPraktikum2\_Kelompok1\_Nama Lengkap}.zip yang berisi file source code (.java) dan laporan praktikum (.pdf).
    \item Deadline pengumpulan: \textbf{29 September 2025}
\end{itemize}


\newpage

\section{Kalkulator Sederhana dengan Operator Aritmatika}
Buat program yang meminta dua bilangan dan sebuah operator (+, -, *, /, \%) kemudian menampilkan hasil operasinya.

\noindent \textbf{Contoh Output:}
\begin{mdframed}[backgroundcolor=gray!15]
\begin{verbatim}
Masukkan bilangan pertama: 10
Masukkan operator (+, -, *, /, %): *
Masukkan bilangan kedua: 5
Hasil: 50
\end{verbatim}
\end{mdframed}

\section{Konversi Suhu (Celsius ke Fahrenheit, Reamur, dan Kelvin)}
Buat program yang menerima input suhu dalam Celsius dan mengonversinya ke Fahrenheit, Reamur, dan Kelvin.

\noindent \textbf{Rumus:}
\begin{align*}
\text{Fahrenheit} &= \text{Celsius} \times \frac{9}{5} + 32 \\
\text{Reamur} &= \text{Celsius} \times \frac{4}{5} \\
\text{Kelvin} &= \text{Celsius} + 273
\end{align*}

\noindent \textbf{Contoh Output:}
\begin{mdframed}[backgroundcolor=gray!15]
\begin{verbatim}
Masukkan suhu dalam Celsius: 25
25°C = 77°F
25°C = 20°R
25°C = 298K
\end{verbatim}
\end{mdframed}

\section{Program dengan Operator Ternary}
Buat program yang menerima dua bilangan dan menggunakan operator ternary untuk menampilkan bilangan yang lebih besar.

\noindent \textbf{Contoh Output:}
\begin{mdframed}[backgroundcolor=gray!15]
\begin{verbatim}
Masukkan bilangan pertama: 8
Masukkan bilangan kedua: 12
Bilangan yang lebih besar adalah: 12
\end{verbatim}
\end{mdframed}

\newpage
\section{Penggunaan Operator Increment dan Decrement}
Buat program yang menunjukkan perbedaan antara prefix dan postfix increment/decrement.

\noindent \textbf{Contoh Output:}
\begin{mdframed}[backgroundcolor=gray!15]
\begin{verbatim}
Nilai awal a = 5
Setelah a++: 5
Setelah ++a: 7
Setelah a--: 7
Setelah --a: 5
\end{verbatim}
\end{mdframed}

\vspace{1cm}
\end{document}
