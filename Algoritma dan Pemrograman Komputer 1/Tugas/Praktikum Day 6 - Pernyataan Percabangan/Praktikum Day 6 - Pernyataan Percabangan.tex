\documentclass[12pt]{article}
\usepackage[a4paper, margin=2cm]{geometry}
\usepackage[utf8]{inputenc}
\usepackage{indentfirst}
\usepackage{enumitem}
\usepackage{fancyhdr}
\usepackage{amsmath}
\usepackage{mdframed}
\usepackage{xcolor}
\usepackage{listings}

\lstset{
    language=Java,
    basicstyle=\ttfamily\small,
    keywordstyle=\color{blue},
    commentstyle=\color{gray},
    stringstyle=\color{red},
    numbers=left,
    numberstyle=\tiny,
    stepnumber=1,
    numbersep=5pt,
    backgroundcolor=\color{white},
    showspaces=false,
    showstringspaces=false,
    showtabs=false,
    frame=single,
    tabsize=2,
    captionpos=b,
    breaklines=true,
    breakatwhitespace=true,
    escapeinside={\%*}{*)}
}

\renewcommand{\thesection}{Soal \arabic{section}}

\pagestyle{fancy}
\fancyhf{}
\rhead{TUGAS PRAKTIKUM ALPRO 1}
\lhead{BAB 8}
\rfoot{Halaman \thepage}

\title{TUGAS PRAKTIKUM ALGORITMA DAN PEMROGRAMAN 1}
\author{}
\date{Pertemuan ke-6\\BAB 8\\Pernyataan Percabangan\\Deadline: 04 November 2025 Pukul 18:59}

\begin{document}

\maketitle
\thispagestyle{empty}
\setcounter{page}{0}

\section*{Petunjuk}
\begin{itemize}
    \item Kerjakan semua soal di bawah ini dengan menggunakan bahasa Java.
    \item Berikan penjelasan singkat untuk setiap program dalam komentar kode.
    \item Program harus dapat di-compile dan di-run tanpa error.
    \item Nama file source code (.java) harus sesuai dengan nama class.
    \item Kumpulkan file source code (.java) untuk setiap program dan laporan praktikum (.pdf).
    \item Source code di dalam laporan wajib dilampirkan menggunakan syntax highlighter.
    \item Format laporan praktikum dapat dilihat di myITS Classroom.
    \item Penamaan file laporan praktikum adalah \texttt{LaporanPraktikum6\_Kelompok1\_Nama Lengkap}.pdf.
    \item Hasil pengerjaan dikumpulkan di myITS Classroom dalam satu file (.zip) dengan nama \texttt{LaporanPraktikum6\_Kelompok1\_Nama Lengkap}.zip yang berisi file source code (.java) dan laporan praktikum (.pdf).
    \item Deadline pengumpulan: \textbf{04 November 2025 Pukul 18:59}
\end{itemize}

\newpage

\section{Program Pengecekan Bilangan (Melengkapi Kode)}
Lengkapi program berikut dengan menggunakan pernyataan \textbf{break}, \textbf{continue}, dan \textbf{return} sesuai petunjuk di komentar:

\begin{lstlisting}
import java.util.Scanner;

public class CekBilangan {
    public static void main(String[] args) {
        Scanner input = new Scanner(System.in);
        
        System.out.print("Masukkan bilangan positif: ");
        int num = input.nextInt();
        
        // LENGKAPI 1: Gunakan RETURN jika bilangan <= 0
        %*\colorbox{yellow}{// Tulis kode di sini}*)
        
        System.out.println("Bilangan ganjil 1 sampai " + num + ":");
        for (int i = 1; i <= num; i++) {
            // LENGKAPI 2: Gunakan CONTINUE untuk melewatkan bilangan genap
            %*\colorbox{yellow}{// Tulis kode di sini}*)
            
            System.out.print(i + " ");
            
            // LENGKAPI 3: Gunakan BREAK jika sudah mencetak 10 bilangan
            %*\colorbox{yellow}{// Tulis kode di sini}*)
        }
        
        input.close();
        System.out.println("\nProgram selesai.");
    }
}
\end{lstlisting}

\textbf{Contoh Output:}
\begin{mdframed}[backgroundcolor=gray!15]
\begin{verbatim}
Masukkan bilangan positif: 20
Bilangan ganjil 1 sampai 20:
1 3 5 7 9 11 13 15 17 19 
Program selesai.

Masukkan bilangan positif: -5
Program selesai.
\end{verbatim}
\end{mdframed}

\textbf{Petunjuk:}
\begin{itemize}
    \item RETURN: Keluar dari program jika input tidak valid
    \item CONTINUE: Lewati bilangan genap 
    \item BREAK: Berhenti setelah mencetak 10 bilangan ganjil
\end{itemize}

\newpage

\section{Program Menu dengan Validasi}
Buat program menu sederhana yang menggunakan \textbf{break}, \textbf{continue}, dan \textbf{return} dengan ketentuan:

\begin{itemize}
    \item Program menampilkan menu:
    \begin{enumerate}
        \item Tampilkan bilangan genap
        \item Tampilkan bilangan ganjil  
        \item Keluar
    \end{enumerate}
    \item Fitur validasi:
    \begin{itemize}
        \item Gunakan \textbf{continue} untuk meminta input ulang jika pilihan menu tidak valid
        \item Gunakan \textbf{return} untuk keluar program jika input menu invalid 3 kali
        \item Gunakan \textbf{break} untuk keluar dari loop setelah menampilkan bilangan
    \end{itemize}
    \item Batas maksimal bilangan yang ditampilkan: 20
\end{itemize}

\textbf{Contoh Output:}
\begin{mdframed}[backgroundcolor=gray!15]
\begin{verbatim}
=== PROGRAM MENU BILANGAN ===
1. Tampilkan bilangan genap
2. Tampilkan bilangan ganjil
3. Keluar

Pilihan Anda: 1
Bilangan genap: 2 4 6 8 10 12 14 16 18 20

Pilihan Anda: 5
Input tidak valid! Coba lagi.

Pilihan Anda: 2  
Bilangan ganjil: 1 3 5 7 9 11 13 15 17 19

Pilihan Anda: 3
Terima kasih!
\end{verbatim}
\end{mdframed}

\textbf{Petunjuk Pengerjaan:}
\begin{itemize}
    \item Gunakan while loop untuk menu utama
    \item Gunakan for loop untuk menampilkan bilangan
    \item Hitung jumlah kesalahan input menggunakan counter
    \item Implementasikan ketiga pernyataan percabangan sesuai kebutuhan
\end{itemize}

\end{document}