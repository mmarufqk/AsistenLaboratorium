\documentclass[12pt]{article}
\usepackage[a4paper, margin=2cm]{geometry}
\usepackage[utf8]{inputenc}
\usepackage{indentfirst}
\usepackage{enumitem}
\usepackage{fancyhdr}
\usepackage{amsmath}
\usepackage{mdframed}
\usepackage{xcolor}
\usepackage{listings}

\lstset{
    language=Java,
    basicstyle=\ttfamily\small,
    keywordstyle=\color{blue},
    commentstyle=\color{gray},
    stringstyle=\color{red},
    numbers=left,
    numberstyle=\tiny,
    stepnumber=1,
    numbersep=5pt,
    backgroundcolor=\color{white},
    showspaces=false,
    showstringspaces=false,
    showtabs=false,
    frame=single,
    tabsize=2,
    captionpos=b,
    breaklines=true,
    breakatwhitespace=true,
    escapeinside={\%*}{*)}
}

\renewcommand{\thesection}{Soal \arabic{section}}

\pagestyle{fancy}
\fancyhf{}
\rhead{TUGAS PRAKTIKUM ALPRO 1}
\lhead{BAB 7}
\rfoot{Halaman \thepage}

\title{TUGAS PRAKTIKUM ALGORITMA DAN PEMROGRAMAN 1}
\author{}
\date{Pertemuan ke-5\\BAB 7\\Perulangan\\Deadline: 27 Oktober 2025}

\begin{document}

\maketitle
\thispagestyle{empty}
\setcounter{page}{0}

\section*{Petunjuk}
\begin{itemize}
    \item Kerjakan semua soal di bawah ini dengan menggunakan bahasa Java.
    \item Berikan penjelasan singkat untuk setiap program dalam komentar kode.
    \item Program harus dapat di-compile dan di-run tanpa error.
    \item Nama file source code (.java) harus sesuai dengan nama class.
    \item Kumpulkan file source code (.java) untuk setiap program dan laporan praktikum (.pdf).
    \item Format laporan praktikum dapat dilihat di myITS Classroom.
    \item Penamaan file laporan praktikum adalah \texttt{LaporanPraktikum5\_Kelompok1\_Nama Lengkap}.pdf.
    \item Hasil pengerjaan dikumpulkan di myITS Classroom dalam satu file (.zip) dengan nama \texttt{LaporanPraktikum5\_Kelompok1\_Nama Lengkap}.zip yang berisi file source code (.java) dan laporan praktikum (.pdf).
    \item Deadline pengumpulan: \textbf{27 Oktober 2025}
\end{itemize}

\newpage

\section{Program Generator Pola Bintang}
Buat program yang menghasilkan berbagai pola bintang berdasarkan input n dengan ketentuan berikut:

\begin{itemize}
    \item Program meminta input sebuah bilangan bulat positif \textbf{n} (tinggi pola)
    \item Program akan menampilkan 4 jenis pola bintang:
    \begin{enumerate}
        \item Segitiga siku-siku rata kiri (gunakan FOR loop)
        \item Segitiga siku-siku rata kanan (gunakan WHILE loop)
        \item Segitiga sama sisi (gunakan DO-WHILE loop)
        \item Pola berlian/diamond (gunakan kombinasi nested loops)
    \end{enumerate}
\end{itemize}

\textbf{Contoh Output:}
\begin{mdframed}[backgroundcolor=gray!15]
\begin{verbatim}
Masukkan tinggi pola (n): 4

=== POLA SEGITIGA SIKU-SIKU RATA KIRI ===
*
**
***
****

=== POLA SEGITIGA SIKU-SIKU RATA KANAN ===
   *
  **
 ***
****

=== POLA SEGITIGA SAMA SISI ===
   *
  ***
 *****
*******

=== POLA BERLIAN ===
   *
  ***
 *****
*******
 *****
  ***
   *
\end{verbatim}
\end{mdframed}

\newpage

\section{Program Generator Deret Matematika}
Buat program yang menghasilkan berbagai deret matematika berdasarkan input n dengan ketentuan berikut:

\begin{itemize}
    \item Program meminta input sebuah bilangan bulat positif \textbf{n}
    \item Program akan menampilkan 4 jenis deret matematika:
    \begin{enumerate}
        \item Deret bilangan ganjil sebanyak n suku (gunakan FOR loop)
            \begin{itemize}
                \item \textbf{Contoh}: 1, 3, 5, 7, 9, ... (selisih +2)
                \item \textbf{Rumus suku ke-k}: $a_k = 2k - 1$
            \end{itemize}
        \item Deret bilangan genap sebanyak n suku (gunakan WHILE loop)
            \begin{itemize}
                \item \textbf{Contoh}: 2, 4, 6, 8, 10, ... (selisih +2)
                \item \textbf{Rumus suku ke-k}: $a_k = 2k$
            \end{itemize}
        \item Deret Fibonacci sebanyak n suku (gunakan DO-WHILE loop)
            \begin{itemize}
                \item \textbf{Contoh}: 1, 1, 2, 3, 5, 8, ...
                \item \textbf{Kondisi awal}: $F(1) = 1$, $F(2) = 1$
                \item \textbf{Rumus Rekursif}: $F(n) = F(n-1) + F(n-2)$ untuk $n > 2$
            \end{itemize}
        \item Deret kuadrat sebanyak n suku (bebas pilih loop)
            \begin{itemize}
                \item \textbf{Contoh}: 1, 4, 9, 16, 25, ...
                \item \textbf{Rumus suku ke-k}: $a_k = k^2$
            \end{itemize}
    \end{enumerate}
\end{itemize}

\textbf{Contoh Output:}
\begin{mdframed}[backgroundcolor=gray!15]
\begin{verbatim}
Masukkan nilai n: 6

=== GENERATOR DERET MATEMATIKA ===
Deret ganjil (FOR)   : 1 3 5 7 9 11
Deret genap (WHILE)  : 2 4 6 8 10 12
Deret fibonacci (DO) : 1 1 2 3 5 8
Deret kuadrat        : 1 4 9 16 25 36
\end{verbatim}
\end{mdframed}

\end{document}