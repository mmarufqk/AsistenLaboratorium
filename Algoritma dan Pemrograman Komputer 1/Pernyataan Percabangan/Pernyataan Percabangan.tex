\documentclass{beamer}
\usepackage{graphicx}
\usepackage{xcolor}
\usepackage{colortbl}
\usepackage{listings}
\usepackage{ragged2e}
\usepackage{url}

\renewcommand{\figurename}{Gambar}
\renewcommand{\tablename}{Tabel}

\usetheme{metropolis}
\setbeamertemplate{navigation symbols}{}

\definecolor{codegreen}{rgb}{0,0.6,0}
\definecolor{codegray}{rgb}{0.5,0.5,0.5}
\definecolor{codepurple}{rgb}{0.58,0,0.82}
\definecolor{backcolour}{rgb}{0.95,0.95,0.92}
\definecolor{magenta}{rgb}{0.58,0,0.82}

\lstset{
    language=Java,
    backgroundcolor=\color{backcolour},   
    commentstyle=\color{codegreen},
    keywordstyle=\color{magenta},
    numberstyle=\tiny\color{codegray},
    stringstyle=\color{codepurple},
    basicstyle=\ttfamily\footnotesize,
    breakatwhitespace=false,         
    breaklines=true,                 
    captionpos=b,                    
    keepspaces=true,                 
    numbers=left,                    
    numbersep=5pt,                  
    showspaces=false,                
    showstringspaces=false,
    showtabs=false,                  
    tabsize=2,
    frame=single,
    columns=flexible
}

% Info presentasi
\title{Algoritma dan Pemrograman Komputer 1}
\subtitle{Bab 8: Pernyataan Percabangan}
\author{Aslam Pandu Tasminto -- 5002241025 \\ M. Ma'ruf Qomaruddin Kafi -- 5002241095}
\date{October 28, 2025}
\institute{Departemen Matematika \\ Fakultas Sains dan Analitika Data \\ Institut Teknologi Sepuluh Nopember}

\begin{document}

% Cover
\begin{frame}
\maketitle
\end{frame}

% Daftar Isi
\begin{frame}{Daftar Isi}
  \tableofcontents
\end{frame}

% Section 1: Pengenalan Pernyataan Percabangan
\section{Pengenalan Pernyataan Percabangan}
\begin{frame}{Konsep Pernyataan Percabangan dalam Java}
  \begin{block}{Apa itu Percabangan?}
    \vspace{0.1cm}
    Pernyataan percabangan adalah pernyataan yang digunakan untuk \textbf{mengatur alur eksekusi} program dengan cara mengubah urutan normal eksekusi kode.
    Seperti \textbf{tombol kontrol} dalam program untuk:
    \begin{itemize}
    \vspace{-0.1cm}
      \item \textbf{Berhenti} di tengah proses (\texttt{break})
      \item \textbf{Lompati} langkah tertentu (\texttt{continue}) 
      \item \textbf{Keluar} dari fungsi (\texttt{return})
    \end{itemize}
  \end{block}

  \vspace{-0.2cm}
  \begin{exampleblock}{Analoginya:}
    Bayangkan sedang membaca buku:
    \vspace{-0.2cm}
    \begin{itemize}
      \item \texttt{break} = berhenti baca bab ini
      \item \texttt{continue} = lewati halaman ini
      \item \texttt{return} = tutup buku, selesai membaca
    \end{itemize}
  \end{exampleblock}
\end{frame}

% Section 2: BREAK Statement
\section{BREAK Statement}
\begin{frame}[fragile]{Sintaks dan Konsep BREAK}
  \begin{block}{Fungsi BREAK Statement}
    Menghentikan eksekusi loop (perulangan) atau percabangan (if else dan switch case) secara paksa dan keluar dari blok kode tersebut.
  \end{block}
  
  \begin{table}
    \scriptsize
    \begin{tabular}{p{0.3\textwidth}|p{0.65\textwidth}}
    \textbf{Jenis BREAK} & \textbf{Deskripsi} \\
    \hline
    \rowcolor{gray!20}
    \texttt{break} (unlabeled) & Menghentikan loop/switch terdekat \\
    \rowcolor{white}
    \texttt{break label} (labeled) & Menghentikan loop dengan label tertentu \\
    \end{tabular}
    \caption{Jenis-jenis BREAK Statement}
  \end{table}
\end{frame}

\begin{frame}[fragile]{BREAK Tidak Berlabel (Unlabeled)}
  \begin{exampleblock}{Contoh BREAK dalam FOR Loop}
    \begin{lstlisting}[basicstyle=\ttfamily\scriptsize]
public class BreakUnlabeled {
    public static void main(String[] args) {
        for (int i = 0; i < 5; i++) {
            System.out.println("i = " + i);
            if (i == 2) {
                break;  // Keluar loop saat i = 2
            }
        }
        System.out.println("Loop selesai");
    }
}
    \end{lstlisting}
  \end{exampleblock}

  \vspace{-0.4cm}
  \textbf{Output Program}
    \colorbox{gray!20}{
      \parbox{0.9\textwidth}{
        {\scriptsize
        \texttt{i = 0\\
        i = 1\\
        i = 2\\
        Loop selesai}
        }
      }
    }
\end{frame}

\begin{frame}[fragile]{BREAK Berlabel (Labeled)}
  \begin{exampleblock}{Contoh BREAK dengan Label}
    \begin{lstlisting}[basicstyle=\ttfamily\tiny]
public class BreakLabeled {
    public static void main(String[] args) {
        int x = 35;
        int y = 4;
        int dummy = y;
        
        mulai:
        while(true) {
            if(y > x) {
                System.out.println("Nilai kelipatan y yang mendekati x adalah " + y);
                break mulai;  // Keluar dari blok 'mulai'
            }
            y = y + dummy;
        }
    }
}
    \end{lstlisting}
  \end{exampleblock}
  
  \textbf{Output Program}
    \colorbox{gray!20}{
      \parbox{0.9\textwidth}{
        {\tiny
        \texttt{Nilai kelipatan y yang mendekati x adalah 36}
        }
      }
    }
\end{frame}

% Section 3: CONTINUE Statement
\section{CONTINUE Statement}
\begin{frame}[fragile]{Sintaks dan Konsep CONTINUE}
  \begin{block}{Fungsi CONTINUE Statement}
    Melompati sisa iterasi saat ini dalam loop dan langsung melanjutkan ke iterasi berikutnya.
  \end{block}
  
  \begin{table}
    \scriptsize
    \begin{tabular}{p{0.35\textwidth}|p{0.65\textwidth}}
    \textbf{Jenis CONTINUE} & \textbf{Deskripsi} \\
    \hline
    \rowcolor{gray!20}
    \texttt{continue} (unlabeled) & Melompati iterasi dalam loop terdekat \\
    \rowcolor{white}
    \texttt{continue label} (labeled) & Melompati iterasi dalam loop berlabel \\
    \end{tabular}
    \caption{Jenis-jenis CONTINUE Statement}
  \end{table}
\end{frame}

\begin{frame}[fragile]{CONTINUE Tidak Berlabel (Unlabeled)}
  \begin{exampleblock}{Contoh CONTINUE dalam FOR Loop}
    \begin{lstlisting}[basicstyle=\ttfamily\scriptsize]
public class ContinueUnlabeled {
    public static void main(String[] args) {
        for(int i = 0; i < 5; i++) {
            if(i == 2) {
                continue;  // Lewati saat i = 2
            }
            System.out.println("i = " + i);
        }
    }
}
    \end{lstlisting}
  \end{exampleblock}

  \vspace{-0.2cm}
  \textbf{Output Program}
    \colorbox{gray!20}{
      \parbox{0.9\textwidth}{
        {\scriptsize
        \texttt{i = 0\\
        i = 1\\
        i = 3\\
        i = 4}
        }
      }
    }
\end{frame}

% Section 4: RETURN Statement
\section{RETURN Statement}
\begin{frame}[fragile]{Sintaks dan Konsep RETURN}
  \begin{block}{Fungsi RETURN Statement}
    \vspace{0.2cm}
    Menghentikan eksekusi program atau method.
    \\ Dalam \textbf{main method} return digunakan untuk menghentikan program secara paksa.
    
    \textbf{\alert{Note:}} Return untuk method akan dipelajari di Modul 9 (Fungsi/Method)
  \end{block}
  
  \begin{table}
    \scriptsize
    \begin{tabular}{p{0.35\textwidth}|p{0.65\textwidth}}
    \textbf{Jenis RETURN} & \textbf{Deskripsi} \\
    \hline
    \rowcolor{gray!20}
    \texttt{return} (di main) & Menghentikan program secara paksa \\
    \rowcolor{white}
    \texttt{return value} (di method) & Mengembalikan nilai (Modul 9) \\
    \end{tabular}
    \caption{Jenis-jenis RETURN Statement}
  \end{table}
\end{frame}

\begin{frame}[fragile]{RETURN untuk Menghentikan Program}
\vspace{-0.2cm}
  \begin{exampleblock}{Contoh RETURN di Main Method}
    \begin{lstlisting}[basicstyle=\ttfamily\tiny]
class kembali {
    public static void main (String args[]) {
        System.out.println("Masukan nilai suatu bilangan");
        Scanner baca = new Scanner(System.in);
        int k = baca.nextInt();
        
        if(k % 2 == 0) {
            System.out.println(k + " merupakan bilangan genap");
            return;  // BERHENTI di sini ketika k genap
        }
        
        System.out.println(k + " merupakan bilangan ganjil");
    }
}
    \end{lstlisting}
  \end{exampleblock}

\vspace{-0.2cm}
\textbf{Output Program (input = 4)}
  \colorbox{gray!20}{
    \parbox{0.9\textwidth}{
      {\tiny
      \texttt{Masukan nilai suatu bilangan\\[-0.5em]
      4\\[-0.5em]
      4 merupakan bilangan genap}
      }
    }
  }
\end{frame}

\begin{frame}[fragile]{RETURN untuk Menghentikan Program (Part 2)}
  \begin{exampleblock}{Method check() - Akan dipelajari di Modul 9}
    \begin{lstlisting}[basicstyle=\ttfamily\scriptsize]
    // Method ini akan dipelajari di Modul 9
    public static String check(int k) {
        if(k % 2 == 0) {
            return "genap";
        } else {
            return "ganjil";
        }
    }
}
    \end{lstlisting}
  \end{exampleblock}
  
  \textbf{Output Program (jika input = 7)}
    \colorbox{gray!20}{
      \parbox{0.9\textwidth}{
        {\scriptsize
        \texttt{Masukan nilai suatu bilangan\\
        7\\
        7 merupakan bilangan ganjil}
        }
      }
    }
\end{frame}

% Section 5: Perbandingan BREAK vs CONTINUE vs RETURN
\section{Perbandingan BREAK vs CONTINUE vs RETURN}
\begin{frame}{Perbandingan BREAK vs CONTINUE vs RETURN}
  \begin{table}
    \scriptsize
    \begin{tabular}{p{0.2\textwidth}|p{0.2\textwidth}|p{0.2\textwidth}|p{0.2\textwidth}}
    \textbf{Kriteria} & \textbf{BREAK} & \textbf{CONTINUE} & \textbf{RETURN} \\
    \hline
    \rowcolor{gray!20}
    Efek & Menghentikan loop & Skip iterasi & Keluar method \\
    \rowcolor{white}
    Scope & Loop/switch & Loop saja & Method saja \\
    \rowcolor{gray!20}
    Eksekusi selanjutnya & Keluar loop & Iterasi berikutnya & Pemanggil method \\
    \rowcolor{white}
    Penggunaan & Pencarian, exit condition & Filter data & Early exit, validasi \\
    \rowcolor{gray!20}
    Dapat berlabel & Ya & Ya & Tidak \\
    \rowcolor{white}
    Dipakai di switch & Ya & Tidak & Ya \\
    \end{tabular}
    \caption{Perbandingan BREAK vs CONTINUE vs RETURN Statement}
  \end{table}
\end{frame}

% Section 6: Best Practices
\section{Best Practices}
\begin{frame}{Best Practices Penggunaan Percabangan}
  \begin{alertblock}{Tips dan Rekomendasi}
    \begin{itemize}
      \item \textbf{Gunakan BREAK dengan hati-hati} - Dapat membuat alur program sulit dilacak
      \item \textbf{Hindari BREAK/CONTINUE berlebihan} - Dapat mengurangi readability
      \item \textbf{Pertimbangkan alternatif} - Terkadang kondisi loop yang baik lebih baik daripada BREAK
      \item \textbf{Gunakan CONTINUE untuk skip kondisi} - Lebih jelas daripada nested if
      \item \textbf{Gunakan RETURN untuk early exit} - Efisien untuk validasi input di awal method
      \item \textbf{Hindari multiple return points} - Kecuali untuk early validation yang jelas
    \end{itemize}
  \end{alertblock}
\end{frame}

% Section 7: Contoh Program
\section{Contoh Kasus}
\begin{frame}[fragile]{Studi Kasus 1: Validasi Input}
  \begin{exampleblock}{Validasi Input dengan RETURN Early}
    \begin{lstlisting}[basicstyle=\ttfamily\scriptsize]
public class InputValidator {
    public static void main(String[] args) {
        System.out.println("Masukkan usia:");
        Scanner input = new Scanner(System.in);
        int age = input.nextInt();
        
        if(age < 0) {
            System.out.println("Usia tidak valid");
            return;  // Hentikan program
        }
        if(age > 150) {
            System.out.println("Usia tidak realistis");
            return;  // Hentikan program
        }
        
        System.out.println("Usia valid: " + age);
    }
}
    \end{lstlisting}
  \end{exampleblock}
\end{frame}

\begin{frame}[fragile]{Studi Kasus 2: Pencarian Karakter}
  \begin{exampleblock}{Pencarian dengan BREAK}
    \begin{lstlisting}[basicstyle=\ttfamily\scriptsize]
public class CariKarakter {
    public static void main(String[] args) {
        String teks = "programming";
        char target = 'g';
        
        for(int i = 0; i < teks.length(); i++) {
            if(teks.charAt(i) == target) {
                System.out.println("Karakter '" + target + "' ditemukan di index: " + i);
                break;  // Berhenti setelah ketemu pertama
            }
        }
    }
}
    \end{lstlisting}
  \end{exampleblock}
\end{frame}

% Section 8: Latihan
\section{Latihan}
\begin{frame}{Latihan 1: Bilangan Prima dengan BREAK}
  \begin{block}{Soal 1: Cek Bilangan Prima}
    Buat program yang mengecek apakah suatu bilangan prima. Gunakan \textbf{BREAK} untuk mengoptimasi proses pengecekan.
  \end{block}
\end{frame}

\begin{frame}{Latihan 2: Skip Bilangan Genap}
  \begin{block}{Soal 2: Cetak Bilangan Ganjil dengan CONTINUE}
    Buat program yang mencetak bilangan ganjil dari 1 sampai 20. Gunakan \textbf{CONTINUE} untuk melewatkan bilangan genap.
  \end{block}
\end{frame}

% Section 9: Kesimpulan
\section{Kesimpulan}
\begin{frame}{Kesimpulan}
  \begin{alertblock}{Inti Bab 8: Pernyataan Percabangan}
    \begin{itemize}
      \item \textbf{BREAK}: Menghentikan perulangan/struktur kontrol secara paksa
      \item \textbf{CONTINUE}: Melompati iterasi saat ini dan lanjut ke iterasi berikutnya
      \item \textbf{RETURN}: Keluar dari . Untuk method non main bisa dengan nilai kembalian
    \end{itemize}
  \end{alertblock}
\end{frame}

% Section 10: Referensi
\section{Referensi}
\begin{frame}{Referensi}
  \begin{block}{Referensi:}
    \begin{itemize}
      \item \textbf{Modul Praktikum Algoritma dan Pemrograman - Modul 8}\\
            Departemen Matematika FSAD ITS
      \item \textbf{Oracle Java Tutorials - Branching Statements}\\
            \url{https://docs.oracle.com/javase/tutorial/java/nutsandbolts/branch.html}
      \item \textbf{GeeksforGeeks - Java Break and Continue}\\
            \url{https://www.geeksforgeeks.org/break-and-continue-statement-in-java/}
      \item \textbf{TutorialsPoint - Java Methods}\\
            \url{https://www.tutorialspoint.com/java/java_methods.htm}
    \end{itemize}
  \end{block}
\end{frame}

% Penutup
\begin{frame}[standout]
  \Huge \textbf{Terima Kasih}
\end{frame}

\end{document}