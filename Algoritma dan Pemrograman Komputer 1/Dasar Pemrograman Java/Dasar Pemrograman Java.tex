\documentclass{beamer}
\usepackage{graphicx}
\usepackage{verbatim}


\usetheme{metropolis}
\setbeamertemplate{navigation symbols}{}
\usepackage{listings}
\usepackage{xcolor}
\usepackage{ragged2e}

\definecolor{codegreen}{rgb}{0,0.6,0}
\definecolor{codegray}{rgb}{0.5,0.5,0.5}
\definecolor{codepurple}{rgb}{0.58,0,0.82}
\definecolor{backcolour}{rgb}{0.95,0.95,0.92}

\lstset{
    language=Java,
    backgroundcolor=\color{backcolour},   
    commentstyle=\color{codegreen},
    keywordstyle=\color{magenta},
    numberstyle=\tiny\color{codegray},
    stringstyle=\color{codepurple},
    basicstyle=\ttfamily\footnotesize,
    breakatwhitespace=false,         
    breaklines=true,                 
    captionpos=b,                    
    keepspaces=true,                 
    numbers=left,                    
    numbersep=5pt,                  
    showspaces=false,                
    showstringspaces=false,
    showtabs=false,                  
    tabsize=2,
    frame=single,
    columns=flexible
}

\addtobeamertemplate{block begin}{}{\justifying}
\addtobeamertemplate{block begin}{}{\vspace{3px}}

% Info presentasi
\title{Pengantar Algoritma dan Pemrograman}
\subtitle{Bab 2: Dasar Pemrograman Java}
\author{Aslam Pandu Tasminto -- 5002241025 \\ M. Ma'ruf Qomaruddin Kafi -- 5002241095}
\institute{Departemen Matematika \\ Fakultas Sains dan Analitika Data \\ Institut Teknologi Sepuluh Nopember}

% Logo untuk title page
\titlegraphic{%
  \includegraphics[height=0.9cm]{Dasar Pemrograman Java/logoprovikom.jpg}%
  \hspace{0.5em}%
  \includegraphics[height=0.9cm]{Dasar Pemrograman Java/logomatematika.png}%
  \hspace{0.5em}%
  \includegraphics[height=1cm]{Dasar Pemrograman Java/logoits.png}%
}

\begin{document}

% Cover
\maketitle

% Daftar Isi
\begin{frame}{Daftar Isi}
  \tableofcontents
\end{frame}

% Section 1: Pengenalan Java
\section{Pengenalan Java}
\begin{frame}{Apa itu Java?}
  \begin{block}{Definisi}
    Java adalah bahasa pemrograman berorientasi objek yang dikembangkan oleh Sun Microsystems sejak tahun 1991.
  \end{block}
  \begin{block}{Pengertian Sederhana}
    Java bisa dianggap sebagai “bahasa untuk memberi instruksi ke komputer” yang mudah dipelajari dan dapat dipakai di berbagai sistem operasi (Windows, Linux, Mac) serta banyak digunakan untuk membuat aplikasi desktop, web, IOS hingga Android.
      \begin{itemize}
        \item Mirip dengan C++ dan Smalltalk
        \item Dirancang agar mudah dipakai dan \textbf{platform independent}
        \item Aman dan portabel untuk pemrograman Internet
      \end{itemize}
  \end{block}
\end{frame}


% Section 2: Platform Independent
\section{Platform Independent}
\begin{frame}{Platform Independent}
  \begin{itemize}
    \item Program Java dapat berjalan di berbagai sistem operasi dan arsitektur komputer
    \item Sifat ini berlaku untuk \textbf{source code} dan \textbf{binary code}
    \item Program Java dijalankan menggunakan \textbf{Java Runtime Environment (JRE)}
    \item Untuk menulis dan mengompilasi program, dibutuhkan \textbf{Java Development Kit (JDK)}
  \end{itemize}
  \begin{center}
  \end{center}
\end{frame}

% Section 3: JDK dan JRE
\section{JDK dan JRE}
\begin{frame}{JDK dan JRE}
  \begin{itemize}
    \item \textbf{JDK (Java Development Kit)}: paket lengkap untuk pengembangan aplikasi Java, berisi \textit{compiler}, \textit{interpreter}, debugger, dan tools lainnya.
    \item \textbf{JRE (Java Runtime Environment)}: lingkungan eksekusi untuk menjalankan program Java, berisi JVM + library standar (tanpa compiler).
    \item Dapat diunduh gratis dari \texttt{https://www.oracle.com/java/technologies/downloads/}
  \end{itemize}
  \begin{block}{Singkatnya\\}
    \textbf{JDK → untuk membuat program Java},  
    \textbf{\\JRE → untuk menjalankan program Java}.
  \end{block}
\end{frame}

% Section 4: Library Java
\section{Library Java}
\begin{frame}{Library Java}
  \begin{itemize}
    \item Memiliki library yang luas untuk:
    \begin{itemize}
      \item Grafik dan user interface
      \item Kriptografi dan keamanan
      \item Jaringan dan database
      \item Suara dan multimedia
    \end{itemize}
    \item Mempermudah pengembangan aplikasi dengan cepat
  \end{itemize}
\end{frame}

% Section 5: Object-Oriented Programming
\section{Object-Oriented Programming}
\begin{frame}{Object-Oriented Programming (OOP)}
  \begin{itemize}
    \item Cara berpikir dalam pemrograman dengan memodelkan dunia nyata sebagai \textbf{objek}.
    \item Setiap objek punya \textbf{data} (atribut) dan \textbf{perilaku} (method).
    \item Membantu membuat program lebih \textbf{terstruktur}, \textbf{mudah dipelihara}, dan \textbf{dapat digunakan kembali}.
    \item Konsep penting: \textit{class}, \textit{object}, \textit{encapsulation}, \textit{inheritance}, \textit{polymorphism}.
  \end{itemize}
\end{frame}

% Section 6: Sintaks Dasar Java
\section{Sintaks Dasar Java}

\begin{frame}[fragile]{Aturan Penamaan}
  \begin{itemize}
    \item \textbf{Case Sensitivity}: \texttt{Hello} ≠ \texttt{hello}
    \item \textbf{Konvensi Penamaan yang Umum Digunakan:}
    \begin{itemize}
      \item \textbf{PascalCase}: untuk nama class. Contoh:(\texttt{HelloMyFirstJavaClass})
      \item \textbf{camelCase}: untuk nama method dan variabel. Contoh: (\texttt{main, myMethodName})
      \item \textbf{snake\_case}: untuk konstanta dan variabel (\texttt{umur\_saya})
      \item \textbf{kebab-case}: tidak direkomendasikan di Java
    \end{itemize}
    \item \textbf{Nama File}: harus sama dengan nama class + \texttt{.java}
  \end{itemize}
\end{frame}

\begin{frame}[fragile]{Struktur Program Java}
\begin{lstlisting}
public class Hello {
    public static void main(String[] args) {
        System.out.println("Hello World");
    }
}
\end{lstlisting}
  \begin{itemize}
    \item Setiap program harus memiliki method \texttt{main}
    \item File harus disimpan sebagai \texttt{Hello.java}
  \end{itemize}
\end{frame}

\begin{frame}[fragile]{Class (Kelas)}
  \begin{itemize}
    \item Setiap program Java harus memiliki minimal satu \textbf{class}.
    \item Nama class harus sesuai dengan nama file.
    \item Konvensi: diawali huruf kapital.
  \end{itemize}
\begin{lstlisting}
public class HelloWorld {
    // kode program ditulis di dalam class
}
\end{lstlisting}
  \small File harus disimpan sebagai: \texttt{HelloWorld.java}
\end{frame}

\begin{frame}[fragile]{Method (Fungsi)}
  \begin{itemize}
    \item Blok kode yang melakukan tugas tertentu.
    \item Method utama wajib: \texttt{public static void main(String[] args)}
    \item Program Java dieksekusi dari method \texttt{main}.
  \end{itemize}
\begin{lstlisting}
public class HelloWorld {
    public static void main(String[] args) {
        // kode yang dijalankan
    }
}
\end{lstlisting}
\end{frame}

\begin{frame}[fragile]{System.out.println()}
  \begin{itemize}
    \item Digunakan untuk mencetak teks ke konsol.
    \item \texttt{System} → kelas built-in Java
    \item \texttt{out} → objek output standar
    \item \texttt{println()} → mencetak baris baru
  \end{itemize}
\begin{lstlisting}
System.out.println("Hello World!");
System.out.print("Tanpa baris baru");
System.out.printf("Format: %s", "teks");
\end{lstlisting}
  \textbf{Output:}
\begin{lstlisting}
Hello World!
Tanpa baris baruFormat: teks
\end{lstlisting}
\end{frame}

\begin{frame}[fragile]{Komentar}
  \begin{itemize}
    \item Komentar menambah catatan dalam kode.
    \item Tidak mempengaruhi eksekusi program.
    \item Ada 3 jenis komentar:
  \end{itemize}
\begin{lstlisting}
// Komentar satu baris

/*
Komentar multi baris
*/

/**
 * Komentar dokumentasi (JavaDoc)
 * @author Nama
 * @version 1.0
 */
\end{lstlisting}
\end{frame}

\begin{frame}[fragile]{Contoh Program Lengkap}
\begin{lstlisting}
/**
 * Program pertama untuk mencetak pesan
 */
public class HelloWorld {

    // Method utama
    public static void main(String[] args) {
        System.out.println("Hello World!");
        System.out.print("Selamat belajar ");
        System.out.println("Java!");
    }
}
\end{lstlisting}
\textbf{Output:}
\begin{lstlisting}
Hello World!
Selamat belajar Java!
\end{lstlisting}
\end{frame}

\begin{frame}[fragile]{Aturan Penting dan Kesalahan Umum}
  \begin{block}{✓ Aturan Penting}
    \begin{enumerate}
      \item Case Sensitivity: \texttt{Hello} ≠ \texttt{hello}
      \item Nama class harus sama dengan nama file
      \item Method \texttt{main} wajib ada
      \item Titik koma (\texttt{;}) wajib di akhir statement
      \item Kurung kurawal (\{\}) untuk blok kode
    \end{enumerate}
  \end{block}

  \begin{alertblock}{Kesalahan Umum}
  \vspace{5px}
\begin{lstlisting}
public class hello { // ✗ huruf kecil
    public static void main(String[] args) {
        System.out.println("Hello") // ✗ tanpa ;
    } // ✗ kurung kurawal tidak balance
}
\end{lstlisting}
  \end{alertblock}
\end{frame}

% Section 7: Kesimpulan
\section{Kesimpulan}
\begin{frame}{Kesimpulan}
  \begin{alertblock}{Inti Bab 2}
    \begin{itemize}
      \item Java adalah bahasa pemrograman berorientasi objek yang dapat dijalankan di berbagai platform (platform independent).
      \item \textbf{JDK} digunakan untuk mengembangkan program, sedangkan \textbf{JRE} digunakan untuk menjalankannya.
      \item Penulisan program harus mengikuti aturan penamaan dan struktur dasar Java.
      \item Library Java sangat luas dan membantu pengembangan aplikasi dengan lebih mudah.
    \end{itemize}
  \end{alertblock}
\end{frame}


% Penutup
\begin{frame}[standout]
  \Huge \textbf{Terima Kasih} \\[1.5em]
  \Large Pertanyaan dan Diskusi
\end{frame}

\end{document}
