\documentclass{beamer}
\usepackage{graphicx}
\usepackage{verbatim}
\usepackage{listings}
\usepackage{xcolor}
\usepackage{colortbl}

\usetheme{metropolis}
\setbeamertemplate{navigation symbols}{}

\definecolor{codegreen}{rgb}{0,0.6,0}
\definecolor{codegray}{rgb}{0.5,0.5,0.5}
\definecolor{codepurple}{rgb}{0.58,0,0.82}
\definecolor{backcolour}{rgb}{0.95,0.95,0.92}

\lstset{
    language=Java,
    backgroundcolor=\color{backcolour},   
    commentstyle=\color{codegreen},
    keywordstyle=\color{magenta},
    numberstyle=\tiny\color{codegray},
    stringstyle=\color{codepurple},
    basicstyle=\ttfamily\footnotesize,
    breakatwhitespace=false,         
    breaklines=true,                 
    captionpos=b,                    
    keepspaces=true,                 
    numbers=left,                    
    numbersep=5pt,                  
    showspaces=false,                
    showstringspaces=false,
    showtabs=false,                  
    tabsize=2,
    frame=single,
    columns=flexible
}

% Info presentasi
\title{Algoritma dan Pemrograman Komputer 1}
\subtitle{Bab 3: Variabel dan Operator}
\author{Nama Anda \\ NRP Anda}
\institute{Departemen Matematika \\ Fakultas Sains dan Analitika Data \\ Institut Teknologi Sepuluh Nopember}
\date{September 22, 2025}


% Logo untuk title page
\titlegraphic{%
  \includegraphics[height=0.9cm]{../assets/logoprovikom.jpg}%
  \hspace{0.5em}%
  \includegraphics[height=0.9cm]{../assets/logomatematika.png}%
  \hspace{0.5em}%
  \includegraphics[height=1cm]{../assets/logoits.png}%
}

\begin{document}

% Cover
\maketitle

% Daftar Isi
\begin{frame}{Daftar Isi}
  \tableofcontents
\end{frame}

% Section 1: Pengenalan Variabel
\section{Pengenalan Variabel}
\begin{frame}{Apa itu Variabel?}
  \begin{block}{Definisi}
    Variabel adalah satuan dasar penyimpanan dalam Java yang bersifat sementara.
  \end{block}
  \begin{itemize}
    \item Menyimpan data yang dapat berubah selama program berjalan
    \item Data dapat berupa bilangan, huruf, atau tipe data lainnya
    \item Memiliki nama (identifier) dan tipe data
  \end{itemize}
\end{frame}

\begin{frame}[fragile]{Identifier (Penamaan Variabel)}
  \begin{block}{Aturan Penamaan Identifier}
    \begin{itemize}
      \item Tidak boleh mengandung spasi
      \item Harus diawali dengan: karakter Unicode, \textcolor{red}{\$} (dollar), atau \_ (underscore)
      \item Case-sensitive (huruf besar dan kecil berbeda)
      \item Tidak dibatasi panjang maksimum
      \item Tidak boleh menggunakan kata kunci Java
    \end{itemize}
  \end{block}
  \begin{exampleblock}{Contoh yang Benar}
    \texttt{nama, nilai1, nama\_saya, \_alamat, \textcolor{red}{\$harga}, \textcolor{red}{nama\$baru}}
  \end{exampleblock}
  \begin{alertblock}{Contoh yang Salah}
    \texttt{1nilai, nama lengkap, class, \textcolor{red}{\$}}
  \end{alertblock}
\end{frame}

\begin{frame}{Identifier (Penamaan Variabel)}
    \begin{alertblock}{Perhatian!}
    Meskipun diperbolehkan, penggunaan \textcolor{red}{\$} tidak disarankan karena:
    \begin{itemize}
      \item Compiler Java menggunakan \$ untuk nama class inner
      \item Dapat membingungkan dan mengurangi readability
      \item Konvensi pemrograman Java tidak menganjurkannya
    \end{itemize}
  \end{alertblock}
\end{frame}

% Section 2: Tipe Data Dasar
\section{Tipe Data Dasar}
\begin{frame}[fragile]{Tipe Data Dasar Java}
  \begin{table}
  \scriptsize
  \begin{tabular}{|>{\columncolor{blue!10}}l|c|>{\raggedright\arraybackslash}p{3cm}|l|}
  \hline
  \rowcolor{blue!20}
  \textbf{Tipe Data} & \textbf{Panjang Bit} & \textbf{Range} & \textbf{Default} \\
  \hline
  \cellcolor{green!10}boolean & 16 & true/false & false \\
  \cellcolor{green!10}char & 16 & 0 to $2^{16}-1$ & \textbackslash u0000 \\
  \cellcolor{green!10}byte & 8 & $-2^7$ to $2^7-1$ & 0 \\
  \cellcolor{green!10}short & 16 & $-2^{15}$ to $2^{15}-1$ & 0 \\
  \cellcolor{green!10}int & 32 & $-2^{31}$ to $2^{31}-1$ & 0 \\
  \cellcolor{green!10}long & 64 & $-2^{63}$ to $2^{63}-1$ & 0L \\
  \cellcolor{green!10}float & 32 & 3.4e-038 to 3.4e+038 & 0.0F \\
  \cellcolor{green!10}double & 64 & 1.7e-308 to 1.7e+308 & 0.0 \\
  \hline
  \rowcolor{yellow!20}
  \cellcolor{yellow!10}String & Variabel & Unicode characters & null \\
  \hline
  \end{tabular}
  \end{table}
  
  \begin{block}{Keterangan}
    \begin{itemize}
      \item \textcolor{blue}{Tipe data primitif} (8 jenis pertama) disimpan langsung dalam memori
      \item \textcolor{orange}{String} adalah tipe data referensi (bukan primitif)
      \item String sebenarnya adalah class yang mewakili urutan karakter
    \end{itemize}
  \end{block}
\end{frame}

\begin{frame}[fragile]{Contoh Penggunaan String}
  \begin{exampleblock}{Deklarasi dan Inisialisasi String\\}
\begin{lstlisting}
// Menggunakan literal string
String nama = "Budi Santoso";

// Menggunakan constructor
String alamat = new String("Jl. Merdeka No. 123");

// Concatenation (penggabungan string)
String salam = "Halo, " + nama + "!";

// Method pada String
int panjang = nama.length();
String kapital = nama.toUpperCase();
String kecil = nama.toLowerCase();
\end{lstlisting}
  \end{exampleblock}
\end{frame}

\begin{frame}[fragile]{Deklarasi dan Inisialisasi Variabel}
  \begin{block}{Sintaks Dasar}  
    \texttt{tipeData namaVariabel = nilai;}
  \end{block}
  
  \begin{exampleblock}{Contoh}
\begin{lstlisting}
int umur = 20;
double harga = 15000.50;
char grade = 'A';
boolean isActive = true;
String nama = "Budi";
\end{lstlisting}
  \end{exampleblock}
\end{frame}

\begin{frame}[fragile]{Operasi pada String}
  \begin{exampleblock}{Beberapa Operasi String yang Umum\\}
\begin{lstlisting}
String str1 = "Hello";
String str2 = "World";
// Penggabungan string
String result = str1 + " " + str2; // "Hello World"
// Perbandingan string
boolean isEqual = str1.equals(str2); // false
// Substring
String sub = result.substring(0, 5); // "Hello"
// Pengecekan prefix/suffix
boolean startsWithH = str1.startsWith("H"); // true
boolean endsWithD = str2.endsWith("d"); // true
// Mengganti karakter
String replaced = str1.replace('l', 'p'); // "Heppo"
\end{lstlisting}
  \end{exampleblock}
\end{frame}

% Section 3: Operator
\section{Operator dalam Java}
\begin{frame}{Jenis-Jenis Operator}
  \begin{itemize}
    \item \textbf{Binary Operator}: 2 operand (contoh: \texttt{+, -, *, /})
    \item \textbf{Unary Operator}: 1 operand (contoh: \texttt{++, --})
    \item \textbf{Ternary Operator}: 3 operand (contoh: \texttt{? :})
  \end{itemize}
\end{frame}

\begin{frame}[fragile]{Operator Aritmatika}
  \begin{table}
  \scriptsize
  \begin{tabular}{|>{\columncolor{blue!10}}l|l|>{\raggedright\arraybackslash}p{4cm}|}
  \hline
  \rowcolor{blue!20}
  \textbf{Operator} & \textbf{Penggunaan} & \textbf{Deskripsi} \\
  \hline
  \cellcolor{green!10}+ & a + b & Penjumlahan \\
  \cellcolor{green!10}- & a - b & Pengurangan \\
  \cellcolor{green!10}* & a * b & Perkalian \\
  \cellcolor{green!10}/ & a / b & Pembagian \\
  \cellcolor{green!10}\% & a \% b & Modulus (sisa bagi) \\
  \hline
  \end{tabular}
  \end{table}

  \vspace{0.3cm}
  \begin{exampleblock}{Contoh Kode}
    \begin{lstlisting}
int a = 15, b = 4;
System.out.println("Penjumlahan: " + (a + b));      // 19
System.out.println("Pembagian: " + (a / b));        // 3 (integer division)
System.out.println("Modulus: " + (a % b));          // 3 (sisa bagi)
    \end{lstlisting}
  \end{exampleblock}
\end{frame}

\begin{frame}[fragile]{Operator Increment dan Decrement}
  \begin{table}
  \scriptsize
  \begin{tabular}{|l|l|l|}
  \hline
  \textbf{Operator} & \textbf{Penggunaan} & \textbf{Deskripsi} \\
  \hline
  ++ & a++ & Menambah nilai a setelah operasi \\
  ++ & ++a & Menambah nilai a sebelum operasi \\
  -- & a-- & Mengurangi nilai a setelah operasi \\
  -- & --a & Mengurangi nilai a sebelum operasi \\
  \hline
  \end{tabular}
  \end{table}
  
  \begin{exampleblock}{Contoh}
\begin{lstlisting}
int x = 5;
System.out.println(x++); // Output: 5
System.out.println(++x); // Output: 7
\end{lstlisting}
  \end{exampleblock}
\end{frame}

\begin{frame}{Operator Relasional dan Logika}
  \begin{columns}[T]
    \begin{column}{0.48\textwidth}
      \begin{block}{Operator Relasional}
        \scriptsize
        \begin{tabular}{|l|l|}
        \hline
        \textbf{Operator} & \textbf{Keterangan} \\
        \hline
        == & Sama dengan \\
        != & Tidak sama dengan \\
        > & Lebih dari \\
        >= & Lebih dari sama dengan \\
        < & Kurang dari \\
        <= & Kurang dari sama dengan \\
        \hline
        \end{tabular}
      \end{block}
    \end{column}
    
    \begin{column}{0.48\textwidth}
      \begin{block}{Operator Logika}
        \scriptsize
        \begin{tabular}{|l|l|}
        \hline
        \textbf{Operator} & \textbf{Keterangan} \\
        \hline
        \& & AND \\
        \textbar & OR \\
        \^{} & XOR \\
        ! & NOT \\
        \&\& & AND (short circuit) \\
        \textbar\textbar & OR (short circuit) \\
        \hline
        \end{tabular}
      \end{block}
    \end{column}
  \end{columns}
\end{frame}

\begin{frame}[fragile]{Operator Kondisi (Ternary Operator)}
  \begin{block}{Sintaks}
    \texttt{kondisi ? nilaiJikaTrue : nilaiJikaFalse}
  \end{block}
  
  \begin{exampleblock}{Contoh}
\begin{lstlisting}
int a = 8, b = 10;
int kecil = a < b ? a : b;
System.out.println("Nilai yang lebih kecil: " + kecil);
\end{lstlisting}
  \end{exampleblock}
  
  \begin{alertblock}{Output}
    \texttt{Nilai yang lebih kecil: 8}
  \end{alertblock}
\end{frame}

\begin{frame}[fragile]{Operator Kombinasi}
  \begin{block}{Operator Gabungan}
    Menggabungkan operasi aritmatika dengan assignment
  \end{block}
  
  \begin{exampleblock}{Contoh}
\begin{lstlisting}
int x = 10;
x += 5;  // x = x + 5 → 15
x -= 3;  // x = x - 3 → 12
x *= 2;  // x = x * 2 → 24
x /= 4;  // x = x / 4 → 6
x %= 5;  // x = x % 5 → 1
\end{lstlisting}
  \end{exampleblock}
\end{frame}

% Section 4: Contoh Program
\section{Contoh Program}
\begin{frame}[fragile]{Contoh: Nilai Default Variabel}
\begin{lstlisting}
public class DefaultValue {
    static boolean b;
    static char c;
    static byte bt;
    static short s;
    static int i;
    static long l;
    static float f;
    static double d;
    
    public static void main(String args[]) {
        System.out.println("Default value b = " + b);
        System.out.println("Default value c = " + c);
        System.out.println("Default value bt = " + bt);
        // ... dan seterusnya
    }
}
\end{lstlisting}
\end{frame}

\begin{frame}[fragile]{Contoh: Operator Increment-Decrement}
\begin{lstlisting}
class IncDec {
    public static void main (String args[]) {
        int x = 8, y = 13;
        System.out.println("x = " + x);
        System.out.println("y = " + y);
        System.out.println("x = " + ++x);  // Pre-increment
        System.out.println("y = " + y++);  // Post-increment
        System.out.println("x = " + x--);  // Post-decrement
        System.out.println("y = " + --y);  // Pre-decrement
    }
}
\end{lstlisting}
\end{frame}

% Section 5: Latihan dan Tugas
\section{Latihan}
\begin{frame}{Latihan}
  \begin{enumerate}
    \item Buat program yang menghitung operasi aritmatika pada a = 100 dan b = 10
    \item Buat program yang menentukan nilai logika dengan c = true dan d = false
    \item Buat program yang menghitung luas dan volume tabung menggunakan Math.PI
    \item Buat program yang menentukan apakah a merupakan modulus dari b menggunakan operator kondisi
  \end{enumerate}
\end{frame}

% Section 6: Kesimpulan
\section{Kesimpulan}
\begin{frame}{Kesimpulan}
  \begin{alertblock}{Inti Bab 3}
    \begin{itemize}
      \item Variabel adalah tempat penyimpanan data yang memiliki nama dan tipe data
      \item Java memiliki 8 tipe data dasar dengan karakteristik yang berbeda
      \item Operator digunakan untuk memanipulasi nilai dan variabel
      \item Pemahaman variabel dan operator adalah dasar pemrograman Java
    \end{itemize}
  \end{alertblock}
\end{frame}

% Penutup
\begin{frame}[standout]
  \Huge \textbf{Terima Kasih} \\[1.5em]
  \Large Pertanyaan dan Diskusi
\end{frame}

\end{document}