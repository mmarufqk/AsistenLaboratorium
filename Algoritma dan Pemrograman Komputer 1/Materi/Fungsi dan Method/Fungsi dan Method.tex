\documentclass{beamer}
\usepackage{graphicx}
\usepackage{xcolor}
\usepackage{colortbl}
\usepackage{listings}
\usepackage{ragged2e}
\usepackage{url}

\renewcommand{\figurename}{Gambar}
\renewcommand{\tablename}{Tabel}

\usetheme{metropolis}
\setbeamertemplate{navigation symbols}{}

\definecolor{codegreen}{rgb}{0,0.6,0}
\definecolor{codegray}{rgb}{0.5,0.5,0.5}
\definecolor{codepurple}{rgb}{0.58,0,0.82}
\definecolor{backcolour}{rgb}{0.95,0.95,0.92}
\definecolor{magenta}{rgb}{0.58,0,0.82}

\lstset{
    language=Java,
    backgroundcolor=\color{backcolour},   
    commentstyle=\color{codegreen},
    keywordstyle=\color{magenta},
    numberstyle=\tiny\color{codegray},
    stringstyle=\color{codepurple},
    basicstyle=\ttfamily\footnotesize,
    breakatwhitespace=false,         
    breaklines=true,                 
    captionpos=b,                    
    keepspaces=true,                 
    numbers=left,                    
    numbersep=5pt,                  
    showspaces=false,                
    showstringspaces=false,
    showtabs=false,                  
    tabsize=2,
    frame=single,
    columns=flexible
}

% Info presentasi
\title{Algoritma dan Pemrograman Komputer 1}
\subtitle{Bab 9: Fungsi/Method}
\author{Aslam Pandu Tasminto -- 5002241025 \\ M. Ma'ruf Qomaruddin Kafi -- 5002241095}
\date{November 2024}
\institute{Departemen Matematika \\ Fakultas Sains dan Analitika Data \\ Institut Teknologi Sepuluh Nopember}

% Logo untuk title page
\titlegraphic{%
  \includegraphics[height=0.9cm]{Fungsi dan Method/logoprovikom.jpg}%
  \hspace{0.5em}%
  \includegraphics[height=0.9cm]{Fungsi dan Method/logomatematika.png}%
  \hspace{0.5em}%
  \includegraphics[height=1cm]{Fungsi dan Method/logoits.png}%
}

\begin{document}

% Cover
\begin{frame}
\maketitle
\end{frame}

% Daftar Isi
\begin{frame}{Daftar Isi}
  \tableofcontents
\end{frame}

% Section 1: Pengenalan Fungsi/Method
\section{Pengenalan Fungsi/Method}
\begin{frame}{Konsep Dasar Fungsi/Method}
  \begin{block}{Apa itu Fungsi/Method?}
    \vspace{0.1cm}
    Fungsi/Method adalah kumpulan beberapa pernyataan yang dikelompokkan untuk melakukan suatu operasi tertentu.
    \begin{itemize}
      \item Seperti \textbf{mesin kecil} dalam program yang melakukan tugas spesifik
      \item Dapat dipanggil berulang kali dari berbagai bagian program
      \item Meningkatkan \textbf{reusability}, \textbf{readability}, \textbf{modularity}, dan \textbf{maintainability} kode
    \end{itemize}
  \end{block}

  \vspace{-0.2cm}
  \begin{alertblock}{Konsep Fundamental Method}
  \scriptsize
    Dalam memahami method, terdapat 3 konsep fundamental yang harus dikuasai:
    \begin{enumerate}
      \item \textbf{Parameter} - Input data yang diproses method
      \item \textbf{Statement} - Logika dan operasi yang dilakukan  
      \item \textbf{Return Type/Value} - Output hasil pemrosesan
    \end{enumerate}
    \textbf{Penting:} Meskipun secara teknis parameter bisa kosong dan return type bisa void, ketiga konsep ini merupakan fondasi pemahaman method.
  \end{alertblock}
\end{frame}

% Section 2: Struktur Fungsi/Method
\section{Struktur Fungsi/Method}
\begin{frame}[fragile]{Anatomi Fungsi/Method}
  \begin{block}{6 Komponen Utama Fungsi/Method}
    \begin{enumerate}
      \item \textbf{Modifier} - Akses kontrol (\texttt{public static})
      \item \textbf{Return Type} - Tipe data kembalian
      \item \textbf{Nama Method} - Identifier fungsi
      \item \textbf{Parameter} - Data input method
      \item \textbf{Exception} - Penanganan error
      \item \textbf{Statement} - Badan method
    \end{enumerate}
  \end{block}

  \begin{exampleblock}{Contoh Struktur}
\begin{lstlisting}[basicstyle=\ttfamily\scriptsize]
public static boolean isBigger(int a, int b) {
    if (a > b) {
        return true;    // Return Value
    } else { 
        return false;   // Return Value
    }
}
\end{lstlisting}
  \end{exampleblock}
\end{frame}

% Section 3: Aturan Penamaan
\section{Aturan Penamaan}
\begin{frame}{Konvensi Penamaan Method}
  \begin{table}
    \scriptsize
    \begin{tabular}{p{0.3\textwidth}|p{0.65\textwidth}}
    \textbf{Pattern} & \textbf{Contoh dan Penjelasan} \\
    \hline
    \rowcolor{gray!20}
    \texttt{camelCase} & \texttt{calculateArea(), getUserName()} \\
    \rowcolor{white}
    Awalan huruf kecil & \texttt{make(), processData(), isEmpty()} \\
    \rowcolor{gray!20}
    Kata kerja deskriptif & \texttt{convertToMeter(), validateInput()} \\
    \rowcolor{white}
    Konsisten & \texttt{getX(), setX(), isX(), hasX()} \\
    \end{tabular}
    \caption{Konvensi Penamaan Method yang Direkomendasikan}
  \end{table}

  \begin{alertblock}{Best Practice}
    \begin{itemize}
      \item Gunakan nama yang menjelaskan \textbf{apa yang dilakukan} method
      \item Hindari nama singkat yang tidak jelas (\texttt{proc(), calc()})
      \item Untuk boolean, gunakan awalan \texttt{is}, \texttt{has}, \texttt{can}
    \end{itemize}
  \end{alertblock}
\end{frame}

% Section 4: Jenis-jenis Method
\section{Jenis-jenis Method}
\begin{frame}[fragile]{Method Tanpa Parameter}
  \begin{exampleblock}{Method Tanpa Input Parameter}
\begin{lstlisting}[basicstyle=\ttfamily\scriptsize]
class MethodSederhana {
    public static void sapa() {
        System.out.println("Halo! Selamat belajar Java!");
    }
    
    public static void main(String[] args) {
        sapa();  // Memanggil method
        sapa();  // Bisa dipanggil berkali-kali
    }
}
\end{lstlisting}
  \end{exampleblock}

  \textbf{Output Program}
    \colorbox{gray!20}{
      \parbox{0.9\textwidth}{
        {\scriptsize
        \texttt{Halo! Selamat belajar Java!\\
        Halo! Selamat belajar Java!}
        }
      }
    }
\end{frame}

\begin{frame}[fragile]{Method Dengan Parameter}
  \begin{exampleblock}{Method dengan Input Parameter}
\begin{lstlisting}[basicstyle=\ttfamily\scriptsize]
class MethodParameter {
    public static void sapaNama(String nama) {
        System.out.println("Halo, " + nama + "!");
    }
    
    public static void hitungLuas(int panjang, int lebar) {
        int luas = panjang * lebar;
        System.out.println("Luas: " + luas);
    }
    
    public static void main(String[] args) {
        sapaNama("Budi");
        hitungLuas(5, 3);
    }
}
\end{lstlisting}
  \end{exampleblock}

    \vspace{-0.3cm}
  \textbf{Output Program}
    \colorbox{gray!20}{
    \scriptsize
      \parbox{0.9\textwidth}{
        {\scriptsize
        \texttt{Halo, Budi!\\
        Luas: 15}
        }
      }
    }
\end{frame}

\begin{frame}[fragile]{Method Dengan Return Value}
\vspace{-0.25cm}
  \begin{exampleblock}{Method yang Mengembalikan Nilai}
\begin{lstlisting}[basicstyle=\ttfamily\scriptsize]
public class MethodReturn {
    public static int tambah(int a, int b) {
        return a + b;
    }
    
    public static boolean adalahGenap(int angka) {
        return angka % 2 == 0;
    }
    
    public static void main(String[] args) {
        int hasil = tambah(10, 5);
        boolean cek = adalahGenap(hasil);
        
        System.out.println("10 + 5 = " + hasil);
        System.out.println(hasil + " genap? " + cek);
    }
}
\end{lstlisting}
  \end{exampleblock}

    \vspace{-0.65cm}
  \textbf{Output Program}
  \tiny
    \colorbox{gray!20}{
      \parbox{0.9\textwidth}{
        {\scriptsize
        \texttt{10 + 5 = 15\\
        15 genap? false}
        }
      }
    }
\end{frame}

\begin{frame}[fragile]{Method Rekursif}
  \begin{exampleblock}{Contoh Method Rekursif (Faktorial)}
\begin{lstlisting}[basicstyle=\ttfamily\scriptsize]
class MethodRekursif {
    public static int faktorial(int n) {
        if (n == 0 || n == 1) {
            return 1;
        } else {
            return n * faktorial(n - 1);
        }
    }
    
    public static void main(String[] args) {
        int angka = 5;
        int hasil = faktorial(angka);
        System.out.println(angka + "! = " + hasil);
    }
}
\end{lstlisting}
  \end{exampleblock}

  \textbf{Output Program}
    \colorbox{gray!20}{
      \parbox{0.9\textwidth}{
        {\scriptsize
        \texttt{5! = 120}
        }
      }
    }
\end{frame}


% Section 5: Parameter vs Argument
\section{Parameter vs Argument}
\begin{frame}[fragile]{Perbedaan Parameter dan Argument}
  \begin{table}
    \scriptsize
    \begin{tabular}{p{0.45\textwidth}|p{0.45\textwidth}}
    \textbf{Parameter} & \textbf{Argument} \\
    \hline
    \rowcolor{gray!20}
    Variabel dalam deklarasi method & Nilai aktual yang diberikan saat pemanggilan \\
    \rowcolor{white}
    \texttt{String nama} dalam deklarasi & \texttt{"Budi"} saat memanggil method \\
    \rowcolor{gray!20}
    Seperti variabel yang didapat & Seperti nilai yang diisi \\
    \rowcolor{white}
    Ditentukan saat menulis method & Diberikan saat menggunakan method \\
    \end{tabular}
    \caption{Perbedaan Konseptual Parameter dan Argument}
  \end{table}

  \begin{exampleblock}{Contoh}
\begin{lstlisting}[basicstyle=\ttfamily\scriptsize]
// Parameter: 'nama' dan 'umur'
public static void perkenalan(String nama, int umur) {
    System.out.println("Nama: " + nama + ", Umur: " + umur);
}

// Argument: "Alice" dan 20
perkenalan("Alice", 20);
\end{lstlisting}
  \end{exampleblock}
\end{frame}

% Section 6: Method Overloading
\section{Method Overloading}
\begin{frame}[fragile]{Konsep Method Overloading}
\vspace{-0.2cm}
  \begin{block}{Apa itu Overloading?}
    Membuat beberapa method dengan \textbf{nama sama} tetapi \textbf{parameter berbeda} (jumlah atau tipe data).
  \end{block}

\vspace{-0.1cm}
  \begin{exampleblock}{Contoh Overloading}
\begin{lstlisting}[basicstyle=\ttfamily\scriptsize]
public class Calculator {
    // Versi 1: dua parameter integer
    public static int tambah(int a, int b) {
        return a + b;
    }
    
    // Versi 2: tiga parameter integer  
    public static int tambah(int a, int b, int c) {
        return a + b + c;
    }
    
    // Versi 3: parameter double
    public static double tambah(double a, double b) {
        return a + b;
    }
}
\end{lstlisting}
  \end{exampleblock}
\end{frame}

\begin{frame}[fragile]{Penggunaan Method Overloading}
  \begin{exampleblock}{Implementasi dalam Program}
\begin{lstlisting}[basicstyle=\ttfamily\scriptsize]
public class Calculator {
    public static void main(String[] args) {
        // Memanggil versi berbeda berdasarkan parameter
        int hasil1 = tambah(5, 3);
        int hasil2 = tambah(1, 2, 3);
        double hasil3 = tambah(2.5, 3.7);
        
        System.out.println("5 + 3 = " + hasil1);
        System.out.println("1 + 2 + 3 = " + hasil2);
        System.out.println("2.5 + 3.7 = " + hasil3);
    }
}
\end{lstlisting}
  \end{exampleblock}

  \textbf{Output Program}
    \colorbox{gray!20}{
      \parbox{0.9\textwidth}{
        {\scriptsize
        \texttt{5 + 3 = 8\\
        1 + 2 + 3 = 6\\
        2.5 + 3.7 = 6.2}
        }
      }
    }
\end{frame}

% Section 7: Studi Kasus
\section{Studi Kasus}
\begin{frame}[fragile]{Studi Kasus 1: Kalkulator}
\vspace{-0.3cm}
  \begin{exampleblock}{Implementasi Kalkulator dengan Method}
  \vspace{-0.15cm}
\begin{lstlisting}[basicstyle=\ttfamily\tiny]
public class Kalkulator {
    public static double tambah(double a, double b) {
        return a + b;
    }
    
    public static double kurang(double a, double b) {
        return a - b;
    }
    
    public static double kali(double a, double b) {
        return a * b;
    }
    
    public static double bagi(double a, double b) {
        if (b == 0) {
            System.out.println("Error: Pembagian dengan nol!");
            return 0;
        }
        return a / b;
    }
    
    public static void main(String[] args) {
        System.out.println("10 + 5 = " + tambah(10, 5));
        System.out.println("10 - 5 = " + kurang(10, 5));
        System.out.println("10 * 5 = " + kali(10, 5));
        System.out.println("10 / 5 = " + bagi(10, 5));
    }
}
\end{lstlisting}
  \end{exampleblock}
\end{frame}

\begin{frame}[fragile]{Studi Kasus 2: Validasi Data}
  \begin{exampleblock}{Method untuk Validasi Input}
\begin{lstlisting}[basicstyle=\ttfamily\tiny]
public class Validator {
    public static boolean isValidEmail(String email) {
        return email.contains("@") && email.contains(".");
    }
    
    public static boolean isValidAge(int age) {
        return age >= 0 && age <= 150;
    }
    
    public static boolean isStrongPassword(String password) {
        return password.length() >= 8 && 
               !password.equals(password.toLowerCase());
    }
    
    public static void main(String[] args) {
        System.out.println("Email valid? " + isValidEmail("user@example.com"));
        System.out.println("Umur valid? " + isValidAge(25));
        System.out.println("Password kuat? " + isStrongPassword("Pass1234"));
    }
}
\end{lstlisting}
  \end{exampleblock}
\end{frame}

% Section 8: Keuntungan Penggunaan Method
\section{Keuntungan Method}
\begin{frame}{Manfaat Penggunaan Method}
  \begin{table}
    \scriptsize
    \begin{tabular}{p{0.3\textwidth}|p{0.65\textwidth}}
    \textbf{Keuntungan} & \textbf{Deskripsi} \\
    \hline
    \rowcolor{gray!20}
    Reusability & Dapat dipakai berulang tanpa menulis ulang kode \\
    \rowcolor{white}
    Modularity & Memecah program kompleks menjadi bagian kecil \\
    \rowcolor{gray!20}
    Maintainability & Mudah di-maintain dan debug \\
    \rowcolor{white}
    Readability & Kode lebih terstruktur dan mudah dibaca \\
    \rowcolor{gray!20}
    Testing & Mudah di-test secara terpisah \\
    \rowcolor{white}
    Collaboration & Memungkinkan kerja tim yang lebih baik \\
    \end{tabular}
    \caption{Keuntungan Menggunakan Method dalam Pemrograman}
  \end{table}
\end{frame}

% Section 9: Best Practices
\section{Best Practices}
\begin{frame}{Best Practices Penggunaan Method}
  \begin{alertblock}{Tips dan Rekomendasi}
    \begin{itemize}
      \item \textbf{Satu tugas satu method} - Setiap method harus melakukan satu tugas spesifik
      \item \textbf{Nama deskriptif} - Nama method harus jelas menjelaskan fungsinya
      \item \textbf{Parameter minimal} - Gunakan parameter secukupnya, hindari terlalu banyak
      \item \textbf{Return early} - Keluar dari method secepat mungkin jika kondisi terpenuhi
      \item \textbf{Fokus pada tugas} - Method sebaiknya hanya mengolah input menjadi output
      \item \textbf{Dokumentasi} - Berikan komentar untuk method yang kompleks
    \end{itemize}
  \end{alertblock}
\end{frame}

% Section 10: Latihan
\section{Latihan}
\begin{frame}[fragile]{Latihan 1: Method Dasar}
  \begin{block}{Soal 1: Method Sederhana}
    Buat method \texttt{volumeTabung(double r, double t)} yang mengembalikan volume tabung. 
    \textbf{Rumus:} $V = \pi \times r^2 \times t$
    
    \textbf{Contoh:}
    \begin{itemize}
      \item Input: r=7, t=10 → Output: 1539.38 (sekitar)
      \item Gunakan \texttt{Math.PI} untuk nilai π
    \end{itemize}
  \end{block}
\end{frame}

\begin{frame}[fragile]{Latihan 2: Method Boolean}
  \begin{block}{Soal 2: Validasi Bilangan}
    Buat method \texttt{isBilanganPrima(int n)} yang mengembalikan \texttt{true} jika bilangan prima, \texttt{false} jika bukan.
    
    \textbf{Ciri bilangan prima:}
    \begin{itemize}
      \item Hanya bisa dibagi 1 dan dirinya sendiri
      \item Contoh: 2, 3, 5, 7, 11, 13, ...
    \end{itemize}
    
    \textbf{Contoh:}
    \begin{itemize}
      \item Input: 7 → Output: true
      \item Input: 9 → Output: false
    \end{itemize}
  \end{block}
\end{frame}

\begin{frame}[fragile]{Latihan 3: Method Rekursif}
  \begin{block}{Soal 3: Deret Fibonacci}
    Buat method rekursif \texttt{fibonacci(int n)} yang mengembalikan suku ke-n dari deret Fibonacci.
    
    \textbf{Rumus Fibonacci:}
    \begin{itemize}
      \item F(1) = 1, F(2) = 1
      \item F(n) = F(n-1) + F(n-2) untuk n \textgreater 2
    \end{itemize}
    
    \textbf{Contoh:}
    \begin{itemize}
      \item Input: 5 → Output: 5
      \item Input: 7 → Output: 13
    \end{itemize}
  \end{block}
\end{frame}

% Section 11: Kesimpulan
\section{Kesimpulan}
\begin{frame}{Kesimpulan}
  \begin{alertblock}{Inti Bab 9: Fungsi/Method}
    \begin{itemize}
      \item \textbf{Fungsi/Method} adalah blok kode yang melakukan tugas spesifik dan dapat dipanggil berulang
      \item \textbf{Struktur method} terdiri dari 6 komponen: modifier, return type, nama method, parameter, exception, statement
      \item \textbf{Jenis method}: tanpa parameter, dengan parameter, dengan return value, dan rekursif
      \item \textbf{Method overloading} memungkinkan beberapa method dengan nama sama tetapi parameter berbeda
      \item Penggunaan method meningkatkan \textbf{Reusability, Readability, Modularity, dan Maintainability} kode
    \end{itemize}
  \end{alertblock}
\end{frame}

% Section 12: Referensi
\section{Referensi}
\begin{frame}{Referensi}
  \begin{block}{Referensi:}
    \begin{itemize}
      \item \textbf{Modul Praktikum Algoritma dan Pemrograman - Modul 9}\\
            Departemen Matematika FSAD ITS
      \item \textbf{Oracle Java Tutorials - Defining Methods}\\
            \url{https://docs.oracle.com/javase/tutorial/java/javaOO/methods.html}
      \item \textbf{GeeksforGeeks - Methods in Java}\\
            \url{https://www.geeksforgeeks.org/methods-in-java/}
    \end{itemize}
  \end{block}
\end{frame}

\begin{frame}{Referensi}
  \begin{block}{Referensi:}
    \begin{itemize}
      \item \textbf{W3Schools - Java Methods}\\
            \url{https://www.w3schools.com/java/java_methods.asp}
      \item \textbf{Programiz - Java Methods}\\
            \url{https://www.programiz.com/java-programming/methods}
    \end{itemize}
  \end{block}
\end{frame}

% Penutup
\begin{frame}[standout]
  \Huge \textbf{Terima Kasih}
\end{frame}

\end{document}