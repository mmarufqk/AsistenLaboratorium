\documentclass{beamer}
\usepackage{graphicx}
\usepackage{xcolor}
\usepackage{colortbl}
\usepackage{listings}
\usepackage{ragged2e}
\usepackage{url}

\renewcommand{\figurename}{Gambar}
\renewcommand{\tablename}{Tabel}

\usetheme{metropolis}
\setbeamertemplate{navigation symbols}{}

\definecolor{codegreen}{rgb}{0,0.6,0}
\definecolor{codegray}{rgb}{0.5,0.5,0.5}
\definecolor{codepurple}{rgb}{0.58,0,0.82}
\definecolor{backcolour}{rgb}{0.95,0.95,0.92}
\definecolor{magenta}{rgb}{0.58,0,0.82}

\lstset{
    language=Java,
    backgroundcolor=\color{backcolour},   
    commentstyle=\color{codegreen},
    keywordstyle=\color{magenta},
    numberstyle=\tiny\color{codegray},
    stringstyle=\color{codepurple},
    basicstyle=\ttfamily\footnotesize,
    breakatwhitespace=false,         
    breaklines=true,                 
    captionpos=b,                    
    keepspaces=true,                 
    numbers=left,                    
    numbersep=5pt,                  
    showspaces=false,                
    showstringspaces=false,
    showtabs=false,                  
    tabsize=2,
    frame=single,
    columns=flexible
}

% Info presentasi
\title{Algoritma dan Pemrograman Komputer 1}
\subtitle{Bab 10: Array}
\author{Aslam Pandu Tasminto -- 5002241025 \\ M. Ma'ruf Qomaruddin Kafi -- 5002241095}
\date{November 2024}
\institute{Departemen Matematika \\ Fakultas Sains dan Analitika Data \\ Institut Teknologi Sepuluh Nopember}

% Logo untuk title page
\titlegraphic{%
  \includegraphics[height=0.9cm]{Fungsi dan Method/logoprovikom.jpg}%
  \hspace{0.5em}%
  \includegraphics[height=0.9cm]{Fungsi dan Method/logomatematika.png}%
  \hspace{0.5em}%
  \includegraphics[height=1cm]{Fungsi dan Method/logoits.png}%
}

\begin{document}

% Cover
\begin{frame}
\maketitle
\end{frame}

% Daftar Isi
\begin{frame}{Daftar Isi}
  \vspace{0.2cm}
  \tableofcontents
\end{frame}

% Section 1: Pengenalan Array
\section{Pengenalan Array}
\begin{frame}{Konsep Dasar Array}
  \scriptsize
  \begin{block}{Apa itu Array?}
    \vspace{0.1cm}
    Array adalah struktur data yang menyimpan kumpulan elemen dengan \textbf{tipe data yang sama} dalam satu variabel.
    \begin{itemize}
      \item Seperti \textbf{deretan loker} dengan nomor-nomor berurutan
      \item Setiap elemen diakses melalui \textbf{indeks} (dimulai dari 0)
      \item Memudahkan pengelolaan data dalam jumlah banyak
    \end{itemize}
  \end{block}

  \vspace{-0.2cm}
  \begin{alertblock}{Analogi Array}
    Bayangkan array seperti deretan \textbf{loker sekolah}:
    \begin{itemize}
      \item Setiap loker memiliki \textbf{nomor} (indeks)
      \item Isi setiap loker harus \textbf{sejenis} (tipe data sama)
      \item Sekolah tersebut memiliki jumlah loker tertentu (ukuran array)
    \end{itemize}
  \end{alertblock}
\end{frame}

\begin{frame}[fragile]{Tanpa Array vs Dengan Array}
  \scriptsize
  \begin{columns}
    % Kolom Kiri - Tanpa Array
    \begin{column}{0.48\textwidth}
      \begin{block}{Tanpa Array}
        \begin{itemize}
          \item Banyak variabel terpisah
          \item Sulit dikelola
          \item Tidak efisien
        \end{itemize}
      \end{block}
      
      \begin{exampleblock}{Kode Tanpa Array}
\begin{lstlisting}[basicstyle=\ttfamily\tiny]
String pemilik_kotak_1 = "Udin";
String pemilik_kotak_2 = "Joko";
String pemilik_kotak_3 = "Budi";
String pemilik_kotak_4 = "Sari";
String pemilik_kotak_5 = "Rina";
String pemilik_kotak_6 = "Gibran";
// ... dan seterusnya
// untuk 100 kotak?
\end{lstlisting}
      \end{exampleblock}
      
      \begin{alertblock}{Masalah}
        \begin{itemize}
          \item Bagaimana jika ada 1000 kotak?
          \item Sulit melakukan operasi massal
          \item Kode menjadi sangat panjang
        \end{itemize}
      \end{alertblock}
    \end{column}
    
    % Pembatas
    \begin{column}{0.04\textwidth}
      \centering
    \end{column}
    
    % Kolom Kanan - Dengan Array
    \begin{column}{0.48\textwidth}
      \begin{block}{Dengan Array}
        \begin{itemize}
          \item Satu variabel untuk semua
          \item Mudah dikelola
          \item Sangat efisien
        \end{itemize}
      \end{block}
      
      \begin{exampleblock}{Kode Dengan Array}
\begin{lstlisting}[basicstyle=\ttfamily\tiny]
String[] pemilik_kotak = {
  "Udin", "Joko", "Budi", 
  "Sari", "Rina", "Gibran"
};
// Akses mudah dengan index
// pemilik_kotak[0] = "Udin"
// pemilik_kotak[1] = "Joko"
\end{lstlisting}
      \end{exampleblock}
      
      \begin{alertblock}{Keuntungan}
        \begin{itemize}
          \item Dapat menampung ribuan data
          \item Mudah di-loop dan diproses
          \item Kode lebih bersih dan terstruktur
        \end{itemize}
      \end{alertblock}
    \end{column}
  \end{columns}
\end{frame}

% Section 2: Keuntungan Penggunaan Array
\section{Keuntungan Array}
\begin{frame}{Manfaat Penggunaan Array}
  \begin{table}
    \scriptsize
    \begin{tabular}{p{0.3\textwidth}|p{0.65\textwidth}}
    \textbf{Keuntungan} & \textbf{Deskripsi} \\
    \hline
    \rowcolor{gray!20}
    Efisiensi Memory & Menyimpan banyak data dalam satu variabel \\
    \rowcolor{white}
    Akses Random & Dapat mengakses elemen mana saja langsung via index \\
    \rowcolor{gray!20}
    Kode Lebih Rapi & Tidak perlu banyak variabel untuk data sejenis \\
    \rowcolor{white}
    Mudah di-Loop & Dapat diproses secara massal dengan looping \\
    \rowcolor{gray!20}
    Kesesuaian Algoritma & Cocok untuk sorting, searching, dll. \\
    \rowcolor{white}
    Kemudahan Maintenance & Lebih mudah dikelola daripada banyak variabel terpisah \\
    \end{tabular}
    \caption{Keuntungan Menggunakan Array dalam Pemrograman}
  \end{table}
\end{frame}

% Section 3: Deklarasi Array
\section{Deklarasi Array}
\begin{frame}[fragile]{Cara Mendeklarasikan Array}
  \begin{block}{3 Cara Deklarasi Array dalam Java}
    \begin{enumerate}
      \item \textbf{Deklarasi kemudian inisialisasi}
      \item \textbf{Deklarasi dan alokasi memory}
      \item \textbf{Deklarasi dengan nilai langsung}
    \end{enumerate}
  \end{block}

  \begin{exampleblock}{Contoh Deklarasi Array}
\begin{lstlisting}[basicstyle=\ttfamily\scriptsize]
// Cara 1: Deklarasi kemudian inisialisasi
int[] angka;
angka = new int[5];

// Cara 2: Deklarasi dan alokasi memory  
int[] angka = new int[5];

// Cara 3: Deklarasi dengan nilai langsung
int[] angka = {1, 2, 3, 4, 5};
String[] nama = {"Udin", "Joko", "Budi"};
\end{lstlisting}
  \end{exampleblock}
\end{frame}

% Section 4: Mengakses Elemen Array
\section{Mengakses Elemen Array}
\scriptsize
\begin{frame}[fragile]{Cara Mengakses dan Mengubah Elemen Array}
  \begin{block}{Menggunakan Index}
    \begin{itemize}
      \item Index dimulai dari \textbf{0} sampai \textbf{panjang array - 1}
      \item Syntax: \texttt{namaArray[index]}
      \item Dapat membaca dan mengubah nilai elemen
    \end{itemize}
  \end{block}

  \begin{exampleblock}{Contoh Akses Elemen}
\begin{lstlisting}[basicstyle=\ttfamily\scriptsize]
public class AksesArray {
    public static void main(String[] args) {
        int[] nilai = {85, 90, 78, 92, 88};
        
        // Mengakses elemen
        System.out.println("Nilai pertama: " + nilai[0]); // 85
        System.out.println("Nilai ketiga: " + nilai[2]);  // 78
        
        // Mengubah elemen
        nilai[1] = 95; // Mengubah nilai kedua
        System.out.println("Nilai kedua sekarang: " + nilai[1]); // 95
    }
}
\end{lstlisting}
  \end{exampleblock}
\end{frame}

% Section 5: Panjang Array
\section{Panjang Array}
\scriptsize
\begin{frame}[fragile]{Menggunakan Property .length}
  \begin{block}{Property .length}
    \begin{itemize}
      \item Setiap array memiliki property \texttt{.length}
      \item Menghasilkan \textbf{ukuran array} (banyaknya elemen)
      \item Berguna untuk looping dan batas akses array
    \end{itemize}
  \end{block}

\vspace{-0.2cm}
  \begin{exampleblock}{Contoh Penggunaan .length}
\begin{lstlisting}[basicstyle=\ttfamily\tiny]
public class PanjangArray {
    public static void main(String[] args) {
        int[] bilangan = {10, 20, 30, 40, 50};
        String[] buah = {"Apel", "Jeruk", "Mangga"};
        
        System.out.println("Panjang array bilangan: " + bilangan.length); // 5
        System.out.println("Panjang array buah: " + buah.length); // 3
        
        // Looping dengan .length
        for(int i = 0; i < bilangan.length; i++) {
            System.out.println("Elemen " + i + ": " + bilangan[i]);
        }
    }
}
\end{lstlisting}
  \end{exampleblock}
\end{frame}

% Section 6: Inisialisasi Array
\section{Inisialisasi Array}
\begin{frame}[fragile]{Inisialisasi Langsung}

  \begin{exampleblock}{Inisialisasi Langsung}
\begin{lstlisting}[basicstyle=\ttfamily\tiny]
class InisialisasiLangsung {
    public static void main(String[] args) {
        // Inisialisasi dengan nilai langsung
        int[] angka = {15, 25, 30, 45};
        String[] nama = {"Jhonny", "Nando", "Queren"};
        
        System.out.println("Array angka: ");
        for(int i : angka) {
            System.out.print(i + " ");
        }
        
        System.out.println("\nArray nama: ");
        for(String n : nama) {
            System.out.print(n + " ");
        }
    }
}
\end{lstlisting}
  \end{exampleblock}

  \textbf{Output Program}
    \colorbox{gray!20}{
      \parbox{0.9\textwidth}{
        {\tiny
        \texttt{Array angka: 15 25 30 45 \\
        Array nama: Jhonny Nando Queren}
        }
      }
    }
\end{frame}

\begin{frame}[fragile]{Inisialisasi Per Elemen}
  \begin{exampleblock}{Inisialisasi Elemen Secara Individual}
\begin{lstlisting}[basicstyle=\ttfamily\tiny]
class InisialisasiPerElemen {
    public static void main(String[] args) {
        // Deklarasi array
        String[] mahasiswa = new String[3];
        
        // Inisialisasi per elemen
        mahasiswa[0] = "Jhonny";
        mahasiswa[1] = "Nando"; 
        mahasiswa[2] = "Queren";
        
        // Menampilkan semua elemen
        System.out.println("Daftar Mahasiswa:");
        for(int i = 0; i < mahasiswa.length; i++) {
            System.out.println((i+1) + ". " + mahasiswa[i]);
        }
    }
}
\end{lstlisting}
  \end{exampleblock}

  \textbf{Output Program}
    \colorbox{gray!20}{
      \parbox{0.9\textwidth}{
        {\tiny
        \texttt{Daftar Mahasiswa: \\
        1. Jhonny \\
        2. Nando \\ 
        3. Queren}
        }
      }
    }
\end{frame}

% Section 7: Looping dengan Array
\section{Looping dengan Array}
\begin{frame}[fragile]{For Loop dengan Array}
  \begin{exampleblock}{Traditional For Loop}
\begin{lstlisting}[basicstyle=\ttfamily\scriptsize]
public class ForTraditional {
    public static void main(String[] args) {
        int[] nilai = {85, 90, 78, 92, 88};
        
        System.out.println("Daftar Nilai:");
        for(int i = 0; i < nilai.length; i++) {
            System.out.println("Nilai " + (i+1) + ": " + nilai[i]);
        }
        
        // Menghitung rata-rata
        int total = 0;
        for(int i = 0; i < nilai.length; i++) {
            total += nilai[i];
        }
        double rataRata = (double) total / nilai.length;
        System.out.println("Rata-rata: " + rataRata);
    }
}
\end{lstlisting}
  \end{exampleblock}
\end{frame}

\begin{frame}[fragile]{Enhanced For Loop (For-Each)}
  \begin{exampleblock}{For-Each Loop untuk Array}
\begin{lstlisting}[basicstyle=\ttfamily\tiny]
public class ForEachArray {
    public static void main(String[] args) {
        int[] bilangan = {1, 2, 3, 4, 5};
        String[] hari = {"Senin", "Selasa", "Rabu", "Kamis", "Jumat"};
        
        // Enhanced for loop (for-each)
        System.out.print("Bilangan: ");
        for(int angka : bilangan) {
            System.out.print(angka + " ");
        }
        
        System.out.print("\nHari: ");
        for(String h : hari) {
            System.out.print(h + " ");
        }
        
        // Menghitung jumlah elemen > 3
        int count = 0;
        for(int num : bilangan) {
            if(num > 3) count++;
        }
        System.out.println("\nBilangan > 3: " + count + " buah");
    }
}
\end{lstlisting}
  \end{exampleblock}
\end{frame}


% Section 8: Studi Kasus
\section{Studi Kasus}
\begin{frame}[fragile]{Studi Kasus 1: Statistik Nilai}
\vspace{-0.3cm}
  \begin{exampleblock}{Analisis Data Nilai dengan Array}
  \vspace{-0.15cm}
\begin{lstlisting}[basicstyle=\ttfamily\tiny]
public class StatistikNilai {
    public static void main(String[] args) {
        int[] nilai = {85, 90, 78, 92, 88, 76, 95, 82, 79, 91};
        
        // Menghitung total
        int total = 0;
        for(int n : nilai) {
            total += n;
        }
        double rataRata = (double) total / nilai.length;
        
        // Mencari nilai tertinggi dan terendah
        int max = nilai[0];
        int min = nilai[0];
        for(int i = 1; i < nilai.length; i++) {
            if(nilai[i] > max) max = nilai[i];
            if(nilai[i] < min) min = nilai[i];
        }
        
        System.out.println("Statistik Nilai:");
        System.out.println("Rata-rata: " + rataRata);
        System.out.println("Nilai tertinggi: " + max);
        System.out.println("Nilai terendah: " + min);
        System.out.println("Jumlah siswa: " + nilai.length);
    }
}
\end{lstlisting}
  \end{exampleblock}
\end{frame}

\begin{frame}[fragile]{Studi Kasus 2: Pencarian dalam Array}
  \begin{exampleblock}{Mencari Elemen dalam Array}
\begin{lstlisting}[basicstyle=\ttfamily\tiny]
public class PencarianArray {
    public static void main(String[] args) {
        int[] array = {10, 20, 25, 30, 45, 50};
        int target = 30;
        boolean ditemukan = false;
        int index = -1;
        
        // Linear search
        for(int i = 0; i < array.length; i++) {
            if(array[i] == target) {
                ditemukan = true;
                index = i;
                break;
            }
        }
        
        if(ditemukan) {
            System.out.println("Nilai " + target + " ditemukan pada index: " + index);
        } else {
            System.out.println("Nilai " + target + " tidak ditemukan dalam array");
        }
        
        // Menampilkan semua elemen array
        System.out.print("Array: ");
        for(int elemen : array) {
            System.out.print(elemen + " ");
        }
    }
}
\end{lstlisting}
  \end{exampleblock}
\end{frame}

% Section 9: Tambahan Materi dari Modul
\section{Tambahan Materi dari Modul}

\begin{frame}[fragile]{Copy Array dengan System.arraycopy()}
  \scriptsize
  \begin{block}{Konsep System.arraycopy()}
    \begin{itemize}
      \item Method built-in Java untuk menyalin array
      \item Syntax: \texttt{System.arraycopy(src, srcPos, dest, destPos, length)}
      \item Lebih cepat daripada loop manual
    \end{itemize}
  \end{block}
  \vspace{-0.2cm}
  \begin{exampleblock}{Contoh Penggunaan}
\begin{lstlisting}[basicstyle=\ttfamily\tiny]
public class CopyArray {
    public static void main(String[] args) {
        int[] asal = {7, 4, 1, 3, 6, 4, 2};
        int[] tujuan = new int[3];
        
        // Salin 3 elemen dari index 0
        System.arraycopy(asal, 0, tujuan, 0, 3);
        
        System.out.print("Asal: ");
        for(int i : asal) System.out.println(i + " ");
        
        System.out.print("\nTujuan: ");
        for(int i : tujuan) System.out.println(i + " ");
    }
}
\end{lstlisting}
  \end{exampleblock}

  \vspace{-0.2cm}
  \textbf{Output:}
    \colorbox{gray!10}{\texttt{Asal: 7 4 1 3 6 4 2    Tujuan: 7 4 1}}
\end{frame}

\begin{frame}[fragile]{Command Line Arguments (args[])}
  \tiny
  \begin{block}{Konsep args[]}
    \begin{itemize}
      \item Parameter \texttt{String[] args} pada main method
      \item Menerima input dari command line saat menjalankan program
      \item Setiap kata dipisah spasi menjadi elemen array
    \end{itemize}
  \end{block}
  
\vspace{-0.2cm}
  \begin{exampleblock}{Contoh Penggunaan}
\begin{lstlisting}[basicstyle=\ttfamily\tiny]
public class CommandLineArgs {
    public static void main(String[] args) {
        // Run: java CommandLineArgs Senin November 2024
        if(args.length >= 3) {
            System.out.println("Hari: " + args[0]);
            System.out.println("Bulan: " + args[1]);
            System.out.println("Tahun: " + args[2]);
        } else {
            System.out.println("Input: <hari> <bulan> <tahun>");
        }
    }
}
\end{lstlisting}
  \end{exampleblock}

\vspace{-0.3cm}
  \begin{exampleblock}{Contoh Eksekusi di Terminal}
\begin{lstlisting}[basicstyle=\ttfamily\tiny]
# Compile program
javac CommandLineArgs.java

# Run dengan arguments
java CommandLineArgs Senin November 2024
\end{lstlisting}
  \end{exampleblock}

  \textbf{Output di Terminal:}
    \colorbox{gray!10}{\texttt{Hari: Senin \newline Bulan: November \newline Tahun: 2024}}

  \begin{alertblock}{Catatan}
    Fitur ini berguna untuk program yang butuh input awal, tapi dalam praktikum sehari-hari jarang digunakan.
  \end{alertblock}
\end{frame}

% Section 10: Best Practices
\section{Best Practices}
\begin{frame}[fragile]{Best Practices Penggunaan Array}
  \begin{alertblock}{Tips dan Rekomendasi}
    \begin{itemize}
      \item \textbf{Inisialisasi dengan nilai default} - Gunakan loop untuk mengisi array dengan nilai awal
      \item \textbf{Validasi index} - Selalu pastikan index dalam range [0, length-1]
      \item \textbf{Gunakan enhanced for} - Untuk iterasi tanpa perlu mengubah elemen
      \item \textbf{Array.length} - Selalu gunakan property length untuk batas looping
      \item \textbf{Hindari angka langsung} - Jangan tulis angka ukuran array langsung, gunakan variabel
      \item \textbf{Documentasi} - Berikan komentar untuk array yang kompleks
    \end{itemize}
  \end{alertblock}

  \begin{exampleblock}{Contoh Baik}
\begin{lstlisting}[basicstyle=\ttfamily\tiny]
// BAIK: menggunakan variabel untuk ukuran array
int jumlahSiswa = 10;
int[] nilai = new int[jumlahSiswa];

// KURANG BAIK: angka langsung
int[] nilai = new int[10];
\end{lstlisting}
  \end{exampleblock}
\end{frame}

% Section 11: Kesimpulan
\section{Kesimpulan}
\begin{frame}{Kesimpulan}
  \begin{alertblock}{Inti Bab 10: Array}
    \begin{itemize}
      \item \textbf{Array} adalah struktur data untuk menyimpan kumpulan elemen bertipe sama
      \item \textbf{Deklarasi array} dapat dilakukan dengan 3 cara berbeda
      \item \textbf{Elemen array} diakses melalui index (dimulai dari 0)
      \item \textbf{Property .length} memberikan ukuran array
      \item \textbf{Looping dengan array} menggunakan for traditional atau enhanced for
      \item \textbf{System.arraycopy()} untuk menyalin elemen antar array
      \item \textbf{String[] args} untuk command line arguments
      \item Array membuat pengelolaan data dalam jumlah banyak menjadi lebih efisien
    \end{itemize}
  \end{alertblock}
\end{frame}

% Section 12: Latihan
\section{Latihan}
\begin{frame}[fragile]{Latihan 1: Manipulasi Array Dasar}
  \begin{block}{Soal 1: Operasi Array Sederhana}
    Buat program yang:
    \begin{enumerate}
      \item Mendeklarasi array integer dengan 5 elemen
      \item Mengisi array dengan nilai 2, 4, 6, 8, 10
      \item Menampilkan semua elemen array
      \item Menghitung dan menampilkan jumlah semua elemen
      \item Menampilkan elemen awal dan akhir
    \end{enumerate}
  \end{block}
\end{frame}

\begin{frame}[fragile]{Latihan 2: Pencarian dan Modifikasi}
  \begin{block}{Soal 2: Array Processing}
    Buat program yang:
    \begin{enumerate}
      \item Memiliki array string: {"apel", "jeruk", "mangga", "anggur", "pisang"}
      \item Mencari index dari buah "mangga"
      \item Mengganti "anggur" dengan "durian"
      \item Menampilkan array sebelum dan sesudah modifikasi pada point 3.
    \end{enumerate}
  \end{block}
\end{frame}

\begin{frame}[fragile]{Latihan 3: Statistik Array}
  \begin{block}{Soal 3: Analisis Data Array}
    Buat program yang:
    \begin{enumerate}
      \item Menerima sebanyak n input array integer dari user
      \item Menghitung:
      \begin{itemize}
        \item Rata-rata nilai
        \item Banyaknya elemen di atas rata-rata
        \item Banyaknya elemen genap dan ganjil
      \end{itemize}
      \item Menampilkan hasil analisis dalam format yang rapi
    \end{enumerate}
  \end{block}
\end{frame}


% Section 13: Referensi
\section{Referensi}
\begin{frame}{Referensi}
  \begin{block}{Referensi:}
    \begin{itemize}
      \item \textbf{Modul Praktikum Algoritma dan Pemrograman - Modul 10}\\
            Departemen Matematika FSAD ITS
      \item \textbf{Oracle Java Tutorials - Arrays}\\
            \url{https://docs.oracle.com/javase/tutorial/java/nutsandbolts/arrays.html}
      \item \textbf{GeeksforGeeks - Arrays in Java}\\
            \url{https://www.geeksforgeeks.org/arrays-in-java/}
    \end{itemize}
  \end{block}
\end{frame}

\begin{frame}{Referensi}
  \begin{block}{Referensi:}
    \begin{itemize}
      \item \textbf{W3Schools - Java Arrays}\\
            \url{https://www.w3schools.com/java/java_arrays.asp}
      \item \textbf{Programiz - Java Arrays}\\
            \url{https://www.programiz.com/java-programming/arrays}
    \end{itemize}
  \end{block}
\end{frame}

% Penutup
\begin{frame}[standout]
  \Huge \textbf{Terima Kasih}
\end{frame}

\end{document}